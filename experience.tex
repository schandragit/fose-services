\section{Experiences}

Over the past few years, we have worked extensively on taking research innovations from the lab to real-world deployment in IBM Global Business Services. In this section, we highlight some of our experiences.

First of all, there is a fundamental tension between client expectation of the service provider being bold and innovative, and at the same time provide predictable, consistent delivery.  When vendors bid for a contract, the client does want the vendor to bring innovation to impact both cost and quality. Once the project gets going, however, the ground level realities of consistent delivery must be met, and therefore project managers in charge of delivery need to be conservative in their approach.

Any new process or tools causes at least a minor disruption in normal flow of the project, and this makes pushing either a new process or a new tool into project delivery significantly harder.  

To compound matters, a project may move from one vendor to another vendor for various reasons, and for this reason, the client may not want to deviate from ``industry standard'' processes and tools, because it would be difficult to back out of any thing experimental. Clients often prohibit vendors from installing additional software in their IT environment.

The second challenge is that the skill levels of the practitioners vary widely, and moreover, business pressures often requirement to move from project to project. In such a scenario, getting practitioners to enthusiastically learn new tools is difficult.  Moreover, tools in which there is no immediate reward to them --- e.g. entering information in a knowledge sharing system --- are even harder to get adopted.

The research challenge is to design tools in a way that are (a) minimally disruptive and (b) easy to learn. Finally, incentivization schemes, also known as gameification, are an exciting area of research with tremendous implications on service delivery.
