\section{Experiences}
\label{sec:experience}
Over the past few years, we have worked extensively on taking research
innovations from the lab to real-world deployment in IBM Global Business
Services. In this section, we highlight some of our experiences.

First of all, there is a fundamental tension between client expectation of the
service provider being bold and innovative, and at the same time provide
predictable, consistent delivery.  When vendors bid for a contract, the client
does want the vendor to bring innovation to impact both cost and quality. Once
the project gets going, however, the ground level realities of consistent
delivery must be met, and therefore project managers in charge of delivery need
to be conservative in their approach.

Any new process or tool causes at least a minor disruption in normal flow of
the project, and this makes pushing either a new process or a new tool into
project delivery significantly harder.  To compound matters, a project may move
from one vendor to another vendor for various reasons, and for this reason, the
client may not want to deviate from ``industry standard'' processes and tools,
because it would be difficult to back out of any thing experimental. Clients
often prohibit vendors from installing additional software in their IT
environment.

The second challenge occurs when a client already has an in-house tool in the
same space as the innovative tool being introduced by the vendor. In such cases,
displacing the incumbent tool based on merits of improved productivity or lower
cost is hard: the client would have the natural inclination to prefer their tool
over a potential lock-in with the vendor-proprietary tool whose stability in
their IT environment is yet to be proven.

The third issue in deploying an innovative tool is that it must solve the
problem that it is targeted for in a holistic manner in the client's application
landscape. For instance, consider an innovative test-automation tool that has
shown the promise of tremendous productivity improvement and lower costs over
conventional test-automation tools, but that is applicable to web applications
only. A client whose application portfolio includes a mix of web, mainframe, and
packaged applications, might appreciate the tool's many benefits, but would
still be reluctant to adopt it because it addresses the automation problem for
only part of their application portfolio.

The fourth challenge, which we also alluded to in the Introduction, is that
convincing clients of an innovative tool's benefits requires immersive
demonstrations over and over again. Often a client requires multiple
demonstrations and even proof-of-concepts in their IT environment and on their
artifacts, and integration with their existing tool chain. Thus, rolling out an
innovation for the next client might require as much work as it did for the
previous client.

The fifth challenge is that the skill levels of the practitioners vary widely
and, moreover, business pressures often require them to move from project to
project. In such a scenario, getting practitioners to enthusiastically learn new
tools is difficult.  Moreover, tools in which there is no immediate reward to
them---\eg entering information in a knowledge management system---are even
harder to get adopted.

The final challenge is that services companies are not equipped to curate
software tools in house. Thus, a tool developed in the lab, even when it proves
its value, sometimes meets with difficulties because there is no organization to
``own'' and support the tool for long periods of time. In the long term, the
tool can neither stay in the lab nor move, and be owned, outside of it,
heralding its eventual obsolence.

The research challenge is to design tools in a way that are (1)~minimally
disruptive and (2) easy to learn. Incentivization schemes, also known as
gameification, are an exciting area of research with tremendous implications on
service delivery.  Open sourcing may be a way to get tools to be supported in
the community for a long time, though this obviously is at odds with the
vendor's desire to have exclusive, differentiating technology.
