
\subsubsection{Knowledge Creation}

What constitutes useful knowledge in a service company? How can such knowledge
be collected and organized? These are some of the questions that need to be
addressed in knowledge creation. Specifically, we discuss three aspects of
knowledge creation: \textit{crawling}, to collect content from diverse
sources; \textit{parsing}, to translate the content in various formats into a
standard format; and \textit{annotating}, to extract metadata from the content
that can help organize it.

\vskip -5pt
\paragraph*{Crawling} The first activity in building an OKMS is
identifying the data that should be stored in the repository and how to obtain
the data. In general, this activity includes a combination of manually provided
data and automatically crawled data, each of which can have its own peculiar
challenges.

For manual data collection, an organization can require its employees to
contribute learnings and software artifacts to the OKMS. For instance, this
could be done via FAQs, where employees outline solutions to common problem
tickets they have resolved; or, it could be done via postmortem reports, usage
stories, and experience reports~\cite{desouza:2005} created by project members
after the resolution of a key project issue. If a client solution involves
software development, employees can be asked to identify reusable software
components and contribute them, after generalization, to the OKMS.

Manual content creation puts extra burden on the employees, beyond their normal
delivery-related responsibilities. Thus, an interesting problem is how to
motivate employees to contribute high-quality content to the
OKMS~\cite{hendriks1999share}. Incentive mechanisms could include ``badges'' as
is done in open-source forums, such as stack exchange to encourage question and
answer contributions. In a service company, would reputation-building incentives
be sufficient or would monetary or career-growth incentives be
necessary~\cite{bartol2002encouraging}?

%% An organization has the option of mandating that each of it's employees
%% contributes their learnings to the OKMS system. Some examples of manual
%% knowledge creations are: (1) frequently asked question, where employee outlines
%% the typical solutions to some common problem tickets they have resolved (2)
%% postmortem reports, usage stories, experience reports\cite{desouza:2005}, that
%% project members can create after they solve a key issue in the project. To
%% enable solution reuse, services organizations often invest in creating software
%% product families
%% \cite{clements2002software}. However, here it's pre-anticipated what could be
%% some solutions that could be of interest to multiple clients in a particular
%% domain such as healthcare and then those solutions are created with appropriate
%% points for variability built in, so solution can be customized for any client
%% intending to use it.  Once a project is completed, employees are also encouraged
%% to identify reusable components in software they created and share them in the
%% knowledge base. This approach of manual content creation adds extra burden on
%% the employees. An open research challenge is to motivate employees to contribute
%% high quality content in the knowledge base \cite{hendriks1999share}. Open source
%% forums such as stack exchange have experimented with elaborate incentive
%% mechanisms in forms of badges to ensure questions and answer contributions on
%% their forums \cite{vasilescu2014social}. In services organizations are
%% reputation building incentives enough or incentives should translate to monetory
%% benefits and/or career progression /cite{bartol2002encouraging}?

In addition, or as an alternative, to manual content creation, content can be
collected through automated crawling of the artifacts produced in a project. For
example, complete traceability from requirements through code to test cases,
along with relevant content, can be extracted from Application Lifecycle
Management tools. Similarly, information about problem tickets and their
resolutions can be extracted from ticket-management systems.  Account teams
periodically report important metrics, such as ticket volumes, code changes, and
service-improvement actions taken, on the client applications being maintained;
these reports could be automatically pushed into the OKMS. The main challenge
here is extraction of this data from the diversity of tools and technologies used to create software
artifacts and related information across projects.

%% Another approach for content creation is to auto-harvest artifacts produced in
%% the SDLC lifecycle and put these in the knowledge management system. Complete
%% traceability from requirement through code to test cases, along with their
%% content is extractable from Application Lifecycle Management tools and this can
%% act as solution packs to put in the repository. Similarly for ticket resolution,
%% information about problem ticket and it's resolution is extractable from ticket
%% management systems. Account teams periodically report on various important
%% metrics on the applications being maintained in the client landscape such as
%% ticket volumes, code changes, service improvement actions taken. These reports
%% can be auto pushed into the knowledge repository. The main challenge here is
%% handling the diversity of tools and technologies used to create SDLC data across
%% projects.

\vskip -5pt
\paragraph*{Parsing} After data has been pulled from various
sources, the next activity involves parsing different document formats and
templates into a standardized format that can be pushed into the knowledge
repository. The diversity of formats, some of which may be proprietary, makes
automated content parsing a challenging problem and a fruitful research
direction. For instance, a possible approach could be based on writing
model-to-model transforms~\cite{debdoot:2010:scc}.

Another issue is that the traceability information necessary for understanding
the context of how and why an artifact was produced might be missing---for
instance, between traceability code commits and bug reports. This occurs
primarily due to lack of integration between the tools used to create the
artifacts. In such cases, automated analyses must be developed for inferring the
traceability; this is an active area of research~\cite{spanoudakis2005software},
with different types of approaches, such as those based on information
retrieval, traceability rules, special integrators, and inference axioms, being
developed.

%% \paragraph*{Parsing} Once all the requisite data has been pulled from various
%% sources, the main challenge is parsing of the data from various document formats
%% and templates into a standardized format that can be pushed into the knowledge
%% repository. Many of the crawled data is in form of documents in proprietery
%% formats such as Microsoft word, power point slides, visio diagrams, excel
%% sheets. How to automate content extraction from client specific work products
%% into the standard format used by knowledge repository is again a direction for
%% research. One approach is to write model to model transforms where a project
%% admin can specify the mapping \cite{debdoot:2010:scc}. Another challenge is that
%% requisite traceability information that is required to understand complete
%% context of how and why an artifact was produced, might be missing. This is
%% because tools being used to create different SDLC artifacts have not been
%% integrated. There is need to auto infer traceability between artifacts For
%% example, if making a code commit, the developer puts in a comment like "Fixed
%% Bug \#145", then with high confidence the change is to fix "Bug \#145" and
%% traceability edge between code file and bug report should be auto-created.
%% . Tracability inference is an active area of
%% research \cite{spanoudakis2005software}. Some proposed approaches use
%% information retrieval (IR) techniques, others use traceability rules, special
%% integrators, and inference axioms.

In service delivery, the OKMS is populated with data collected at client
engagements and can include client-confidential and sensitive information (\eg a
problem ticket might contain the contact information of the user). Thus, a
critical requirement for the OKMS, which would benefit from automation, is
cleansing or anonymization of all sensitive and client-confidential information.

%% The content in a OKMS database comes from various clients:. There are strict
%% privacy constraints around what data is client confidential and hence cannot be
%% shared. Within a single artifact itself, there might be small portions of
%% content that are client confidential. E.g. a problem ticket might content the
%% contact information of the user who encountered the information. How to remove
%% client confidential data from artifacts put in the repository, how to anonimize
%% the content so as to not disclose client identity and how to ensure that only
%% authorized users and roles have access.

\vskip -5pt
\paragraph*{Annotating}
The content in the OKMS must be organized for easy retrieval. The general
practice to do this is to classify OKMS content against predefined
taxonomies. For example, a simple taxonomy is the artifact type: requirement,
code, problem ticket, etc. More generally, a taxonomy ought to include
information about industry type and process area~\cite{apqc,bph}, technologies
used, etc. Organizations spend effort in building and maintaining these
taxonomies; automated tool support to assist with this would be useful. For
instance, clustering~\cite{Berkhin06} and topic-modeling
techniques~\cite{Blei:2012} could be leveraged to group similar content and
infer topics from them.

Manual classification of data can be tedious and prone to human errors. To
reduce this effort, automation via pattern matching and machine-learning-based
classification approaches~\cite{bishop2006pattern} could be used.  The
effectiveness of these techniques depends on the availability of positive and
negative samples to train a learning model. Often, due to unavailability of
training data and/or lack of differentiating features, usual learning techniques
such as na\"{\i}ve bayes, support vector machines, and decision
trees, do not attain sufficient precision and recall. Thus, there is a
need to customize more advanced learning approaches (\eg
ensemble techniques~\cite{Dietterich:2000}) or develop new techniques that are more effective
on the typical data available in service delivery.

%% One taxonomy is obviously the object type i.e. requirement, code,
%% problem ticket (further segmented into problem description, resolution). Another
%% taxonomy captures the domain the artifact was produced in i.e. industry type and
%% process area e.g. \cite{apqc,bph}. Another taxonomy is the technology
%% used. Organizations spend effort building and maintaining these taxonomies. For
%% every data that is put into the knowledge base, the content needs to be manually
%% categorized against these pre-defined taxonomies. Pattern matching and machine
%% learning based classification approaches /cite{bishop2006pattern} can be used to
%% auto-categorize content to these pre-defined taxonomies. These techniques rely
%% on the availability of equal proportion of positive and negative samples to
%% train a learning model. However, due to unavailability of training data and/or
%% lack of differentiating features, usual learning techniques such as naive bayes,
%% support vector machines, decision trees, might end up not giving desired
%% efficacy (measured as precision and recall). There is need to customize more
%% advanced learning approaches such as adaptive learning, ensemble techniques or
%% develop new techniques that work well with SDLC data. Another interesting area
%% for research is to explore how to help grow the taxonomy over time based on
%% content coming in the knowledge base. Techniques such as
%% clustering \cite{Berkhin06}, topic modeling \cite{Blei:2012} help group together
%% similar looking content and infer topics out of them.


The data in an OKMS can quickly grow very large. For example, in just a year, a
problem ticket repository we setup in IBM grew to contain 750,000 tickets;
similarly, our business process solution repository~\cite{Goodwin:2012b}
contains approximately 16,000 solutions. Given this scale, ensuring high quality
of the data (\eg filtering out non-reusable content) is challenging but
essential. Different approaches could be followed for ensuring data quality,
such as human vetting of incoming OKMS content or based on human feedback on
usefulness of retrieved content. An interesting research direction would be to
explore automatic quality scoring of artifacts, taking into consideration
technical and non-technical artifact content, reputation of artifact authors,
etc. For example, such approaches have been investigated
in~\cite{Majumdar:2011}.

%% Once an OKMS system is implemented in an organization, irrespective of the
%% approach to collect content i.e. manual, automated harvesting or hybrid, the
%% repository starts filling up fast. Over a period of one year, the problem ticket
%% repository we setup within IBM collected 750K tickets. Similarly, the business
%% process solution repository has 16000 solutions. However, not all content being
%% put in the repository is high quality and reusable. Hence, it is becomes
%% imperative to be able to filter out useless content. One way to achieve this is
%% by making a human vet every content being pushed in the repository and only
%% content that is deemed high quality is published. Another approach is to ask
%% people who are retrieving and potentially using the content, give feedback on
%% whether they found content useful or not. The third approach that makes for an
%% interesting research direction is to explore how a an automatic quality score
%% can be assiged to each artifact based on content in the artifact, prior
%% reputation of people who authored the content, whether the project was a success
%% or not and so on. In our problem ticket repository, we experimented with
%% calculating a quality score per ticket based on technical versus non-technical
%% content present in the ticket \cite{Majumdar:2011}.


%% \paragraph*{Summary} To summarize, content creation in OKMS provides multiple
%% oppurtunities where research can contribute. These include: what and how to
%% extract content from SDLC repositories and proprietry document formats, how to
%% infer traceability between different artifacts, how to classify and categorize
%% content, how to maintain privacy, estimate quality and motivate employees to
%% contribute high quality content.
