\section{Globally distributed software development}
In the highly competitive services marketplace, businesses are seeking new ways to take advantage of labor arbitrage using a global resource pool – to get services that are “cheaper, better, faster”.  As the services market place has matured, it has become increasingly apparent that hap-hazard use of outsourcing is a double edged sword – traditional project management techniques do not scale well to a global workforce, unless they are supplemented by an over-arching organizational and procedural structure suitable for supporting distributed teams of practitioners.  

Distributed development and its implicit challenges have become the norm in recent years and this phenomenon has received a lot of interest~\cite{glo24,glo26}. The nature of these challenges has been analyzed in recent literature. Evaristo~\cite{glo27} categorizes them into five critical areas: Perceived distance (geographical and temporal), National culture (languages, accepted work patterns), Development methodology (similarity in processes), Task structure (clarity and structure to team hand-offs) and Organizational distance. The premise is that as the distance between the teams along these critical dimensions increases, the overhead and difficulties associated with distributed development become more prominent.

Collaborative development platforms that promote structured interaction between team members can help make distributed development more efficient~\cite{glo28,glo29}. Sharing expertise and the ability to leverage best practices across engagements is a key part of ensuring efficient collaboration – Expertise Browser~\cite{glo30} and Hipikat~\cite{glo31} are examples of tools that enable sharing at the level of processes and artifacts. Given the rapid churn of resources in the global workforce, a push model of delivering information in situ to developers is important and many of the proposed tools~\cite{glo29,glo31,glo34} support this model.

Formal process modeling, strict enforcement of process controls and monitoring the software development process have long been touted as the right way to streamline global delivery~\cite{glo32,glo33,glo34}. Using capability maturity models as a way to rate the operational maturity of an organization has been in vogue for many years and has been widely adopted by the industry~\cite{glo35}. Milos~\cite{glo34} provides a structured approach to instrumenting processes and tasks.

Reducing process variance through the adoption of techniques to improve IT Service delivery is becoming more popular; see e.g. the LEAN example in the article by Upton~\cite{glo37}. However, translating LEAN techniques to IT faces significant challenges – careful inventory management and improving production line efficiency which form cornerstones of LEAN in a manufacturing setting do not have direct analogues in service delivery. Thus LEAN in the services setting is a tailoring of the four core principles as espoused by Spear and Boen41. These rules are:

\begin{enumerate}
\item{Rule 1: All work shall be highly specified as to content, sequence, timing, and outcome.}
\item{Rule 2: Every customer-supplier connection must be direct, and there must be an unambiguous yes-or-no way to send requests and receive responses.}
\item{Rule 3: The pathway for every product and service must be simple and direct.}
\item{Rule 4: Any improvement must be made in accordance with the scientific method, under the guidance of a teacher, at the lowest possible level in the organization.}
\end{enumerate}

An interesting second-stage update on this case study~\cite{glo38} found that the application of LEAN principles resulted in the adoption of an agile methodology for software development (as opposed to a waterfall approach) and led to significant improvements in quality and productivity. These improvements were attributed to a willingness to quickly share and learn from mistakes, from standardization, and from the promotion of creative problem solving.

subsection{Globally integrated development}
The last few years have seen an accelerated trend towards organizations utilizing a globally integrated development (GID) model for complex IT solution implementation. As with other services, in the first wave of the move towards globally integrated development, the emphasis was on reducing the cost structure through labor arbitrage.  The competition among the major industry players is generating the need for aggressive differentiation that goes beyond the benefits of lower cost –improving quality and reducing time-to-market are becoming key differentiators. In practice, traditional project management techniques that worked well with small co-located teams do not seem to scale well to a global workforce. Unless the development process adopted for GID is supplemented by an over-arching procedural and architectural framework, that implicitly partitions the development process, there is strong evidence to suggest that it will not be effective.

As GID matures a second wave of IT services innovation is leading to the adoption of software development and deployment practices that improve quality and increase productivity while preserving the advantages of labor arbitrage.  The adoption of a disciplined model for delivery by GID has resulted in new organizational efficiencies, reduced process variance and execution discipline. Figure~\ref{glofig1}, shows the vision of the globally integrated enterprise by focusing on one of the key pain points that inhibit this vision – a structured way of distributing work to remote teams that promotes collaboration and enforces good governance practices.  Some of the key organizational entities in this approach include: governance and operations teams, a design center that builds the architecture in support of client requirements, and the technology assembly centers which deliver the work. IBM has been a pioneer in this field -- their approach to structured delivery of IT services by a globally distributed delivery team is called application assembly optimization (AAO)~\cite{gloaao}. Other major IT services companies have adopted variations of this same approach. We describe the generic organization of the software delivery in such a paradigm in the rest of this section.

<<Figure 1 -- A model for globally integrated development in large enterprises>>

Work envelopes are central constructs in this approach to structured delivery of software projects that enable disparate technology assembly centers to come together on-demand in the context of a client engagement.  The work envelopes represent a standard envelope by which every work order is authored, transported, and delivered. Work packets include workflow, instruction (normative guidance), metrics collection, and risk management/exception handling mechanisms. Each work envelope constitutes a sub-set of activities that are bound to a larger project plan WBS managed through a traditional project management (PM) tool. Figure~\ref{glo-fig2} shows an example organization of a work envelope.

<<Figure 2 -- work envelope>>

The goals of organizing delivery in this fashion include offering expert guidance through structured work packets that reduce training time and mentoring, consistent workflow through its governance of process execution and deliverable handoffs, and detailed tracking through timely and accurate windows into task status across globally distributed teams for improved performance and continuous improvement. This approach assumes many of the foundational elements of an organization, and therefore needs governance which integrates into existing governance models at the client I/T shop. 

Most clients already employ a rich set of governance processes. The governance model adopted by the distributed delivery organization described above will augment existing processes with business direction and strategy and ensure that goals and objectives are clearly defined and measured. The Governance team specifies the principles, policies and procedures that are carried out by the Operations teams. The team also aligns IT investments in support of the business directions and defines the IT policies and procedures necessary to meet business goals. Financial and portfolio management are key responsibilities of the Governance team.

The Operations teams manage the daily activities required to meet business goals, while complying with the policies and procedures set forth by the Governance team. Delivery management is a key responsibility of Operations; it includes scheduling and dispatching the work packets to technology assembly centers. Such activities entail managing the delivery resources based on skills, seniority and availability.

Supply and demand planning is critical in such an approach. The Governance team works closely with business champions during their annual budgeting process to understand project needs. Supply and demand projections are adjusted quarterly based on project progress and evolving needs. Project plans include a detailed staffing plan that defines the number of resources, experience, skills, and training required to successfully deliver projects. These plans are maintained, adjusted and communicated by the project manager throughout the life of the project.

The Operations teams regularly update the capacity model, which shows the consumption of committed resources by role for each project, the available capacity, and the anticipated forecast for the year. The Design Center is responsible for collecting and formalizing the business requirements from the lines of business. Analysis of business requirements is also carried out in order to derive associated IT requirements. In turn, these IT requirements are translated into scenarios, use-cases and high-level designs that are validated by the business before they are turned over to the Technology Assembly Centers.

The Design Center is populated with a mix of skills that range from Business Analysts to IT Architects. It is the interaction of these skilled individuals that verifies the alignment of system architecture and business requirements. The Design Center uses work packets that create the processes, input/output data, and work products used in delivering work orders to the technology assembly centers.

The identification, creation, management and reuse of assets are key productivity and quality levers in software development. The last decade has witnessed major advances in transforming the middleware layer of the IT stack into an assortment of standards-based assets. Today, corporations are focused on transforming monolithic applications into more granular, SOA-based assets. The Design Center will manage this transformation and take responsibility for cataloging, indexing, provisioning, versioning and promoting the use of all process and service assets.

Work envelopes enable running multiple, geographically distributed Technology Assembly Centers concurrently to maximize the throughput and realize cost efficiencies. Work envelope scope might involve performing a single development activity on a fully integrated system or a multitude of activities on a single component of a system. Coordinating the activities of Technology Assembly Centers and monitoring the progress of such activities is performed by the Operations teams.



\label{sec:global}



