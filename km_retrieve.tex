\vspace{-5pt}
\subsubsection{Knowledge Retrieval}

Challenges in knowledge retrieval are related to the search capabilities offered
by the OKMS and the effort required from users in determining the usefulness of
the recommended data. We discuss the following aspects of knowledge retrieval:
indexing and searching, search result organization, and advanced analytics.

\vskip -5pt
\paragraph*{Indexing and Searching} The simplest approach for searching
is \textit{keyword-based search}, in which users specify the words they are
looking for and the system returns all artifacts that contain the specified
words. More sophisticated systems use \textit{faceted search}, where the user
can navigate a hierarchical structure (\eg a taxonomy) and select values from
predefined categories.  Despite the availability of many search technologies,
studies such as \cite{idc2}) have shown that retrieval of relevant information
from organizational repositories remains challenging, with users being
successful in less than 50\% of their attempts in searching for information.

Language-based information-retrieval techniques~\cite{manning2008introduction},
on which many knowledge systems are based, may be inadequate in dealing with the
types of repositories we are talking about---that contain not just a collection
of artifacts, but a network of linked artifacts.  Graph databases for storage
and semantic search techniques~\cite{Guha:2003} or extending keyword search for
graphs~\cite{kacholia2005bidirectional} may be more promising in this scenario,
and are worthy of research investigation.

%% \paragraph*{Index and Search} Indexing is how the knowledge repository stores
%% data internally. Search features then work on this index.  Typical ways to
%% retrieve content from a knowledge base are: (1) keyword based search where user
%% specifies a couple of words (s)he is looking for and all artifacts in the
%% repository that contain the words from user query are returned, (2) faceted (or
%% navigational) search where user is shown a hierarchy structure (taxonomy) and
%% can browse information by choosing one or more values from each of the
%% pre-defined categories. Various language based information retrieval
%% models \cite{manning2008introduction} such as vector space models, probablistic
%% models, latent semantic index exist that can be used here. But inspite of easy
%% to use search technologies being available, prior studies report that knowledge
%% retrieval from organization wide repositories remains a challenge. As
%% per \cite{idc,idc2}, while employees spend 15\% to 35\% of their time searching
%% for information in an enterprise, they are successful less than 50\% of the time
%% in finding what they are looking for. Most existing OKMS systems use language
%% based IR models to store and retrieve knowledge. However, as we saw in content
%% creation section, the repository is not just a collection of artifacts but a
%% network of linked artifacts. Use of graph databases for data storage and
%% retrieval techniques such as semantic search techniques \cite{Guha:2003} or
%% optimizing keyword search for graphs \cite{kacholia2005bidirectional}, are worth
%% exploring.

An important factor in performing accurate search is the expressiveness of query
construction. Simple keyword queries, consisting of a set of words, may not be
discriminative enough to return accurate results. A few approaches have been
developed to address this problem, for example, by giving more weights to words
that appear in the title of a problem ticket than words that appear in the
ticket description~\cite{Sinha:2012}, and using separate queries for different
types of information, such as description, application information, and stack
trace, in a problem ticket~\cite{Ashok:2009}.

Another question pertains to assigning weights to clauses in the query while
ranking the search results~\cite{Debdoot:2011:bpm}. Moreover, the idea
of \textit{contextual search}~\cite{wen2004probabilistic,kraft2005q}, which
attempts to capture the user's information needs better by augmenting the query
with contextual information extracted from search context, has shown promising
results. Further research along these directions---in the context of the
knowledge needs in service delivery---investigating different ways of creating
contexts, using contexts in constructing queries, and assigning weights to query
clauses would be interesting.

%% Non-availability of content or poorly organized content can be one reason for
%% this. Another reason could be the inadequacy of the query itself that are used
%% to retrive the content. Suppose a user is trying to find problem tickets that
%% resolved similar issues to what (s)he has been assigned. (S)he would pick up a
%% couple of words from the ticket that describe the problem and use it to query
%% the knowledge base. However, these words might not be discriminative enough and
%% user might end up getting too many or too little search hits. Another approach
%% could be to use complete content in the ticket and use it as query. However in
%% this case, the search engine might end up returning irrelevant results as equal
%% weightage was given to all words in the problem ticket. \cite{Sinha:2012} tries
%% to address this issue by giving more weightage to those words in the query that
%% belong to ticket title. \cite{Ashok:2009} parses out different datatypes from a
%% problem ticket such as description, application information, stack trace and
%% using a different query/search mechanism for each. E.g. instead of just
%% specifying "null pointer exception", the query generated from a problem ticket
%% could be---description: contains following bag of words \{null, pointer,
%% exception\}, process: is equal to "order to cash", stack trace: contains
%% foo.*bar. The results obtained from such a query are likely to be more precise
%% than what a keyword search would yield. A challenge with this approach is how to
%% compose the various clauses in the query---are they "anded" or "ored" or a
%% combination. Another challenge is how much weightage to give to each clause when
%% ranking the search results \cite{Debdoot:2011:bpm}. Moreover, recently
%% contextual search \cite{wen2004probabilistic,kraft2005q} has been gaining
%% momentum. Contextual search tries to better capture a user's information need by
%% augmenting the query with contextual information extracted from search
%% context. It would be an interesting direction for future research to see for
%% each of the knowledge needs in services organizations, what can make up the
%% context, how to use this context to create the query and how to weigh different
%% clauses in the query.

\vskip -5pt
\paragraph*{Search Result Organization}
%% Although more powerful querying capabilities can help improve the accuracy of
%% search results
A search-based system would in general return multiple results. The next
question then is: how much effort does it take for the user to sift through the
results to find relevant information? Simple approaches such as highlighting
matching keywords will not work for complex queries; more sophisticated
techniques that help the user easily comprehend the relevance of the results are
needed. One such approach is \textit{summarization}, which has been used in
text-processing domains to end users get a quick overview of information;
summarization has also been applied to bug
reports~\cite{Mani:2012,Rastkar:2010}. Extending this notion to other types of
artifacts, such as code, requirements, and solution designs, is an open topic
for research.

Another type of analysis that can improve the search results is detection and
grouping of duplicates or near-duplicates. By grouping together such artifacts
and highlighting the variances in seemingly similar artifacts, the system can
reduce information overload on the user.  Detection of duplicate bug reports has
been widely researched
(\eg \cite{wang2008approach,sun2010discriminative}). Developing such analyses
for other types of artifacts, from a services OKMS perspective, would be useful.

Typically, an OKMS would display the search results as a ranked list. The
ranking can be based on relevance of artifacts, quality of artifacts, past user
ratings given to artifacts, etc. Prior research have explored other novel ways
to visualize search results (\eg \cite{Nowell:1996,Shneiderman:2000}).  Another
potential direction for research is the investigation of effective non-list
visualizations of OKMS artifacts.

%% What kinds of non-list visualization
%% would be effective in organization knowledge management system that are
%% predominantly composed of SDLC artifacts is another direction for research.

%% \paragraph*{Search Result Organization}
%% Any search based system would return multiple results. For the user it becomes a
%% chore in itself to go over each of the search results and identify if it is
%% relevant or not. Techniques that can help the user easily comprehend the
%% relevance of the returned results are much needed. Text based search engines
%% highlight the matching words between query and content of the artifact
%% returned. However, when the query becomes complex (as above), keyword
%% highlighting is not of much help. In text processing domain, summarization is
%% one approach that is used to help end users get a quick idea of what the content
%% is about. In \cite{Mani:2012,Rastkar:2010} authors present different techniques
%% to summarize bug reports. Research can help in developing summarization
%% techniques that work for other SDLC artifact such as code, requirements,
%% solution design documents and so on. Further, out of the search results returned
%% many artifacts could be duplicates or near duplicates of each other. In order to
%% reduce information overload for end user, the OKMS system should be able to
%% group together duplicates and near duplicates and be able to highlight the
%% variances in seemingly similar artifacts returned in the search
%% results. Duplicate bug report
%% detection \cite{wang2008approach,sun2010discriminative} has been a widely
%% researched topic. Similarly for each of the artifact of interest from a services
%% OKMS perspective, a customized duplicate detection approach might be needed.

\vskip -5pt
\paragraph*{Advanced Analytics}
Most knowledge-management systems today stop at providing search
capabilities. Given the magnitude of data that can be collected in OKMS, there
is the opportunity of implementing advanced analytics that go beyond by
supporting decision making. Consider the scenario of troubleshooting where, to
resolve a ticket, the user searches through past resolved tickets to find
similar tickets.  Based on the past tickets and the current context, the user
has to formulate potential hypotheses about the root cause, decide which
hypotheses are applicable, and then investigate the solutions. The OKMS would be
much more useful if it could automatically generate potential hypotheses about
root causes and solutions for the user to investigate.

Service requests or tickets contain a lot of information in unstructured text format. Textual analytics applied on this data can help provide useful insights. For example, clustering on service tickets text could help identify problem areas that are causing higher maintanence grief and hence are good candidates for doing preventive work. This approach is discussed in \cite{mani2014}. Analysis of groups of similar problems and resolutions can be used to auto extract out frequently asked question and answers from service requests. In \cite{henss12} authors have presented a text mining and natural language processing based approach to extract FAQs from questions being asked on mailing lists of open source projects.

The effectiveness of knowledge retrieval in the context of service delivery
depends to a large extent on the user's ability to identify projects similar to
the project under consideration. What makes two projects similar? Is it the
technologies being used, the project size, similarity of applications, or a
combination of these and other factors? Investigation of this question and
development of automated techniques for identifying similar projects is needed.

%% \paragraph*{Advanced Analytics}
%% OKMS today stops at providing search capabilities. Given a user query, the
%% system returns back matching artifacts but makes no attempt at making any
%% deductions. Consider the scenario of troubleshooting. A user is searching
%% through the past resolved ticket repository to see if similar issues were
%% resolved. There are multiple reasons why the issues could have arisen. Based on
%% past tickets, the user needs to come up with potential hypothesis of why the
%% issue could be arising and based on conditions (s)he is seeing in the current
%% landscape decide what hypothesis is applicable and then pick the appropriate
%% solution. From such a user's perspective the OKMS system would be more effective
%% if it were able to auto-generate these potential root causes and solutions
%% hypothesis. Consider the service improvement use case, the accout team needs
%% support from OKMS to identify similar projects, then compare the problems seen
%% in the current project with different kind of issues arising in similar
%% projects, then judge whether current team is doing better or worse and then
%% decide to take some action. What kind of capabilities are needed in OKMS system
%% to be useful in above scenarios is another area for research.

%% Much of the knowledge retrieval first requires that user is able to identify
%% similar projects in the organization and then dig deeper into them for learning
%% or comparison with situation at hand. What makes two projects similar? Is it
%% technologies being used or project size or similarity in applications or a
%% combination? Hence, analysis to identify similar projects, compare a project
%% against multiple other projects and auto-summarize similarities and differences
%% is needed.

In general, there are two aspects of organization knowledge management:
codification and personalization~\cite{hansen2000s}. Our discussion of OKMS has
so far focused on the codification aspect, where knowledge is carefully captured
and stored in a knowledge base for use by anyone in the company. The
personalization aspect ties knowledge to specific people; the main purpose of
knowledge management here is to help the user find and connect with people who
have the requisite knowledge, not store the knowledge. This strategy is
especially appropriate in cases where knowledge cannot be easily codified, such
as the knowledge for handling situations that require complex decision
making. Expertise browsing, recommendation, and mining have been explored in
various contexts (\eg \cite{Balog:2006, Mockus:2002}). Extending
these ideas to the context of service delivery, where the OKMS contains data
from multiple projects in different contexts, is an area where further research
can help.

%% There are two techniques for organization knowledge management, codification or
%% personalization \cite{hansen2000s}.  Till now what we have discussed is what is
%% called codification. Here knowledge is carefully captured and stored in the
%% database, where it can be accessed and used by anyone in the company. This
%% strategy allows many people to search for and retrieve knowledge without having
%% to contact the person who originally developed it. This opens up the possibility
%% of achieving required scale in knowledge reuse in services
%% organizations. Another strategy for knowledge management is
%% personalization. Here knowledge is tied to person who developed is and is shared
%% mainly through direct person-to-person contacts. The chief purpose of knowledge
%% management system here is to help people find and connect to other people who
%% have requisite knowledge, not to store it. This strategy works well when
%% knowledge cannot be codified especially knowledge required to handle situations
%% that require complex decision making. Expertise browser \cite{Mockus:2002},
%% expertise recommender \cite{McDonald:2000} have attempted to identify experts on
%% various topics within a project. \cite{Balog:2006} has explored expertise mining
%% and recommendation in predominantly document based OKMS systems. What
%% information to use to mine expertise when repository contains SDLC data from
%% multiple projects, how to match your current context and job profile to suggest
%% people you should have in your network is another scenario where research can
%% help.

%% \paragraph*{Summary} To summarize, content retrieval in OKMS system provides
%% open research problems in query generation and use of context to augment
%% queries, search result summarization, duplicate detection, use of semantic
%% search to improve search relevance. Further the retrieval capabilities need to
%% move beyond just search to provide capabilities such as question-answer
%% generation, cross-account comparison/benchmarking and expertise recommendation.


