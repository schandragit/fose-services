
\section{Research Direction: Testing and Troubleshooting}
\label{sec:testing-debugging}

\subsection{Software Testing as a Service}

Within the increasing demand for software development as a service,
\textit{software testing as a service} has seen significant growth and adoption
in its own right, often as a separate service that is independent of development
activities. In fact, according to a 2006 survey,\footnote{\scriptsize
  \url{http://www.drdobbs.com/architecture-and-design/cheapers-not-always-better/184415486?requestid=247829}}
software testing was the second largest outsourced software-engineering activity
after coding: 81\% of the 200 industrial practitioners, who participated in the
survey, stated that they outsource software testing. Given this trend, services
companies now routinely offer services that focus exclusively on testing
activities.

%% The engagement modes can vary: staff augmentation, core/flex, managed service
%% (fixed capacity, outcome based). But perhaps this needs to be mentioned earlier
%% as these modes are not particular to testing services.

The nature of the activities performed in a testing-services engagement can vary
from client to client; but, typically, the scope of work includes activities,
such as test design, test-data creation, test automation, test execution, and
test maintenance. These activities, of course, pertain to testing in general
(whether performed in-house or in an outsourced manner), but there are factors
that can add unique challenges in the setting of testing services. For instance,
many of these activities can require specific skills in testing techniques/tools
or coding expertise, which the average practitioner involved in service delivery
may not possess---at least not in the myriad of technologies 

%% Limited control over test execution environment

Research Problem: Efficient and change-resilient test automation without good
programming skills or deep tool knowledge (e.g., DOM representations in
tools). Maintenance of test scripts by non-developers.

No one technique of UI element identification is likely to be the best in all
circumstances; therefore, the choice of the technique needs to be based on the
requirements for testing (e.g., execution of scripts across browsers, across
internationalized variants of application, etc.)

Cross-browser testing; keeping up with browser upgrades, reduce reliance on
DOM structure and attributes.

Research direction

Emphasize links to program synthesis: (1) basic generation of change-resilient
method of identifying a UI element; (2) synthesis of custom functions based on
some specification of user intent.

Problem of test automation in the mobile app space. More diversity and
fragmentation. Bigger problem: UI differences across app variants.

Research Problem: Test planning and test design

Research Problem: Test data provisioning. Privacy issues, dependence on client.

Research Problem: Test management and optimization


Work done in IBM, work done by others (in industry and academia), where do we
see this topic going
