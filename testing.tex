
\section{Research Direction: Testing}
\label{sec:testing-debugging}

Within the increasing demand for software development as a service,
\textit{software testing as a service} has seen significant growth and adoption
in its own right, often as a separate service that is independent of development
activities. In fact, according to a 2006 survey,\footnote{\scriptsize
  \url{http://www.drdobbs.com/architecture-and-design/cheapers-not-always-better/184415486?requestid=247829}}
software testing was the second largest outsourced software-engineering activity
after coding: 81\% of the 200 industrial practitioners, who participated in the
survey, stated that they outsource software testing. Given this trend, services
companies now routinely offer services that focus exclusively on testing
activities.

%% The engagement modes can vary: staff augmentation, core/flex, managed service
%% (fixed capacity, outcome based). But perhaps this needs to be mentioned earlier
%% as these modes are not particular to testing services.

\begin{figure}[t]
\centering
\includegraphics[width=\columnwidth, clip, trim = 20mm 133mm 55mm
  18mm]{figs/testing-activities.pdf}
\vspace*{-15pt}
\caption{Typical testing activities performed in delivery of testing services.}
\vspace*{-10pt}
\label{fig:testing-activities}
\end{figure}

The nature of the activities performed in a testing-services engagement can vary
from client to client; but, typically, the scope of work includes activities,
such as test design, test-data creation, test automation, test execution, and
test maintenance. Figure~\ref{fig:testing-activities} presents an overview of
the commonly performed activities in the delivery of testing services. A
delivery engagement might involve any subset of these activities. For example,
delivery for a client might involve only manual test execution, or the scope of
work may include test automation and test maintenance as well. In other cases,
the scope of work may be even broader, encompassing test design and/or test
planning. The nature of the test cases can also vary from functional tests to
tests that focus on validating non-functional system properties, such as
performance and security.

The activities shown in Figure~\ref{fig:testing-activities}, of course, pertain
to testing in general (whether performed in-house or in an outsourced manner),
but there are factors that can add unique challenges in the setting of testing
services. For instance, many of these activities can require specific skills in
testing techniques/tools or coding expertise, which the average testing
practitioner involved in service delivery may not possess. Second, some of the
activities may need to be carried out in strict time-bound cycles and
client-controlled test environments. Finally, the myriad of technologies that
must be accommodated in the contexts of the IT systems of different clients adds
another layer of complexity for service companies.

\subsection{Test Planning}
\label{sec:test-planning}

\subsection{Test Optimization}
\label{sec:test-design}

Optimization of test cases using techniques such as combinatorial test design.

\subsection{Test Data Generation}
\label{sec:test-data}

%\subsection{Test Data Generation}

Enterprise applications are typically database-oriented, in that they make queries over databases and the program flow depends on the data retrieved as a result of a query.  In order to test whether the application satisfies its requirements, sufficient amount of test data must be made available in the database. Consider, for example, a banking application, in which one of the requirement to exercise is that if a customer makes a certain fixed deposit, and the person is a senior citizen, and also holds a checking account in the bank, then an addition 0.25\% is paid as interest. To test that this situation is handled correctly in the application, there should exist a customer in the database such that his or her age qualifies the person as senior citizen and the person also has a checking account. Notice that typically, customer data and account data would be maintained in separate database tables. Typically business requirements are complex, and there need to be specific entries in multiple tables for a scenario to work out.

When a client gives a testing contract, the client may expect the service provider to exercise all the business rules, but may not give enough sample data in the database to be able to exercise these rules.  It then becomes the responsibility of the service provider to figure out the requisite test data to be present in the pertinent tables.

Randomly generated test data  cannot be expected to suffice for enterprise applications with complex rules. Also, systematic test generation approaches based on program analysis cannot be expected to tackle enterprise application, which use a mix of multiple language and database technologies in their implementation. The database community has done some work in creating test data for SQL queries, and that may be relevant; however, as mentioned before, enterprise applications are implemented with a mix of technologies.

We see a big opportunity for research contributions in this subarea. How to take a specification of the application as testing criteria, and automatically populate database that would enable various scenarios that fulfill the test criteria to be exercised.


\subsection{Test Automation}
\label{sec:test-automation}

The functional test cases created during test design are often stated informally
in natural language. These tests are known as \textit{manual test cases}, which
consist of a sequence of steps, intended for execution by a human on the
application user interface.  Test automation is the activity of converting these
test cases into test scripts or programs that perform the test steps
mechanically. After creation, the automated test scripts can be executed in
regression cycles without any human intervention.

Although test automation is desirable (\eg for efficient and predictable test
executions) and sometimes necessary (\eg to meet time-constrained regression
schedules), creating \textit{robust} test scripts that can be used repeatedly in
regression cycles is expensive. In essence, creating such scripts is
time-consuming, involving a significant amount of coding, and requires expertise
in test automation tools (\eg \cite{hpqtp,ibmrft,selenium}). In the context of
service delivery, this poses the unique challenge: how can change-resilient test
scripts be created efficiently by delivery personnel who may not possess good
programming skills or deep tool knowledge?

%% Moreover, test automation involves not only the one-time cost of creating the
%% scripts, but also a recurring cost of maintaining the scripts in response to
%% application changes and changes in the execution environment.

While script robustness is certainly relevant in the context of evolution of the
application under test or its persistent data, it is also pertinent when a
script needs to be executed on different platforms, execution environments, or
application variants. For example, functional testing of web applications needs
to be performed on different browsers to detect potential cross-browser
incompatibility
defects~\cite{Choudhary2010,Shauvik:2012,Choudhary:2013,Mesbah:2011}. Similarly,
in the context of mobile applications, tests for a native mobile application
would need to run the variants of the application for different platforms (\eg
Android, iOS, BlackBerry). Ideally, in such situations, a robust test script
should be agnostic to browser/platform-specific differences unless, of course,
those differences are symptoms of application defects.

\subsubsection*{Research Problem: Tackling Script Fragility}

The core problem that needs to be addressed in attempts to solve the
script-fragility problem is how to locate user-interface (UI) elements in a
reliable manner. In general, there are two broad categories of techniques for
locating UI elements. The first category relies on the structure of the internal
representation the UI~\footnote{\small In the context of web applications, this
  internal representation is the Document Object Model (DOM).} and on internal
attributes, such as IDs or names, of UI elements/widgets. Sample test automation
tools in this category include IBM Rational Functional Tester~\cite{ibmrft}, HP
QuickTest Professional (QTP)~\cite{hpqtp}, and
Selenium~\cite{selenium}. Over-reliance on internal structures and attributes
can make scripts fragile (\eg if the structure changes often or the attributes
are generated dynamically) and, in particular, unreliable for execution across
browsers and platforms.

The second category of tools rely on image processing to locate UI elements;
sample tools in this space include Eggplant~\cite{Eggplant} and
Sikuli~\cite{Chang:2010, Yeh:2009}. These tools record a test script as a
sequence of actions performed on images. During automation, the tester grabs a
portion of the screen around the element of interest, specifies the ``hotspot''
in this portion (the location where, \eg a click action would be performed), and
specifies the image-similarity threshold to be used during test execution to
locate the element. By being independent of internal structure and attributes,
such tools are resilient to changes in the internal structure/attributes, but
they are fragile in the presence of differences in visual rendering, say across
browsers or application variants.

A third, and less-common, alternative to these approaches is to associate with a
UI element a label in the vicinity of that element, and refer to the element via
its associated label. This approach, which is used in the \textsc{ata} tool
developed at IBM Research~\cite{thummalapenta:2012b, thummalapenta:2012a,
  thummalapenta:2013a} and also supported by QTP~\cite{hpqtp}, can overcome the
limitations of approaches that rely on internal structure or image
processing. But, the currently available implementations of these approaches
perform simple label associations (based on visual proximity only) and are
inapplicable when ambiguous labels exist. Development of more powerful and
generally applicable techniques for label association would be a fruitful
direction of research.

These approaches have different strengths and weaknesses; it is quite possible
that no one approach turns out to be the best in all circumstances. For example,
an image-based technique may outperform an internal-structure-based technique
for cross-browser test execution; but, the converse would likely be true for
test execution across internationalized variants of a web application. In
practice, the choice of the technique would have to be tailored to the
requirements of testing.  Development of tools that combine the approaches, and
rigorous empirical studies that evaluate the resiliency of these approaches for
different types of testing would be useful.

%% Cross-browser testing: keeping up with browser upgrades, reduce reliance on DOM
%% structure and attributes.

\subsubsection*{Research Problem: Automated Test Adaptation}

Test automation in the space of mobile applications faces the form\-idable
challenge of high diversity, which occurs along multiple dimensions: many
platforms, devices, web browsers (in the case of mobile web applications), and
application variants. For instance a mobile application can have native, hybrid,
and web variants; moreover, in the case of native and hybrid applications,
different platform-specific and device-specific variants can exist.

Efficiently testing an application across different combinations requires
automation, but the automated test scripts must be self-adaptive. Native or
hybrid mobile applications typically have variants that execute on different
platforms (\eg Android, iOS, and BlackBerry) and different types of devices
(phones or tablets). For some application variants, the UI layout and flow may
be exactly the same, in which case a test script that has been recorded in a
platform-agnostic script notation (\eg \cite{PerfectoScriptOnce}) on one
application variant can be executed, without modification, on another
variant. But, in more complicated cases, the UI layout and flow for the same
scenario may differ across variants---as an archetypal example, the phone and
tablet variants of a mobile application would typically support many common
scenarios that are realized via different sequences of UI events. In such
instances, can a script automated on the tablet variant be automatically adapted
for execution on the phone variant? We believe there is scope for the
development of new techniques that perform such adaptation, while ensuring that
the intent of the test cases are preserved.

%% Problem of test automation in the mobile app space. More diversity and
%% fragmentation. Bigger problem: UI differences across app variants.

\subsubsection*{Research Problem: Automated Test-Script Synthesis}

The problem of converting manual test cases to automated test scripts can be
looked upon as an instance of program synthesis~\cite{Gulwani:2010}.  Automated
program synthesis addresses the problem of discovering a program that realizes a
user intent. There are three dimensions of the synthesis problem: the form of
user intent, the search space of programs, and the search
technique~\cite{Gulwani:2010}. For instance, the user intent could be stated in
different forms such as, natural language, input-output examples, logical
relations between inputs and outputs, and demonstrations. In test automation,
user intent is specified in the manual test steps, and the end goal is an
automated script that realizes that intent.

Recent work~\cite{thummalapenta:2012a} has attempted to automate the creation of
test scripts from manual test steps, using a combination of natural-language
processing and dynamic exploration of multiple flows via backtracking to search
for the correct test script from the space of possible scripts. But, there are
practical limitations to that technique. First, dynamic exploration of
alternative flows in the application UI via backtracking can encounter problems,
such as undoing persistent state updates during an exploration or finding
disable UI elements that limit the scope of exploration. Addressing such
problems is important for effective exploration of the space of possible test
scripts.  Second, test automation can sometimes require the coding of custom
functions; for example, a verification step in a test might require checking
that the values in a drop-down list are sorted or that some value exists in a
particular cell of a table. Currently, such code must be written
manually. Automatic synthesis of such code based on appropriate specification of
user intent would go a long way in automating the overall creation of test
scripts.

\subsection{Test Maintenance}
\label{sec:test-maintenance}

Although developing scripts in a change-resilient manner, as discussed in
Section~\ref{sec:test-automation}, can reduce test-maintenance effort, it cannot
totally eliminate maintenance. In particular, refactoring of the application
GUI, such as splitting a web page into multiple tabbed pages, can break the flow
of a test~\cite{thummalapenta:2013a}, which would occur irrespective of how
robust the mechanism for locating UI elements is. Such changes require the test
flow to be repaired, involving addition or deletion of test
steps---automatically performing such repairs is beyond the capability many of
existing GUI test repair techniques~\cite{Choudhary:2011, Grechanik:2009,
  Memon:2008}. Proposals GUI-refactoring-driven test repair~\cite{Daniel2011}
accommodate certain types of GUI refactorings (\eg replacing a radio box with a
drop-down list), but not general flow repair.  Recent research has seen the
development of repair techniques for broken GUI flows~\cite{Zhang2013},
addressing some of the limitations of the older test-repair techniques, but
there are opportunities for more research contributions on this topic.




