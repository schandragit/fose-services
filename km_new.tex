
\section{Knowledge Management}
\label{sec:km}

Software development is inherently a knowledge-intensive activity. Software
designers and developers leverage their software-development skills, along with
domain knowledge, past experiences, and the knowledge of team members, to solve
the problem at hand, such as implementing a new feature or resolving a bug.  In
small, collocated teams, knowledge management is not a big challenge---people's
expertise on different parts of a system is typically known. New team members
can use informal communication channels to identify experts and seek their help
as needed.

However, as teams increase in size or become geographically distributed,
knowledge management starts to become challenging. In large or distributed
teams, system knowledge---\eg expertise, dependencies, best practices---is
spread across multiple people, locations, and (in the case of outsourcing) even
organizations~\cite{Desouza:2006}. In such projects, a knowledge-management
system is needed to create \textit{project memories} that can serve different
needs, such as assisting new team members in identifying experts to reach out to
for specific questions, or helping existing team members determine the relevant
artifacts, and the people they might need to coordinate with, for performing a
task. Existing attempts at creating such project memories include
Hipikat~\cite{Murphy:2005} and Codebook~\cite{Begel:2010}.

%% First, it should assist new team members in understanding the project with
%% little or no face-to-face guidance and identifying experts to reach out to for
%% questions. Second, the system should help existing team members identify the
%% artifacts relevant, and the people they might need to coordinate with, for
%% performing a task. Existing attempts at creating such project memories include
%% Hipikat~\cite{Murphy:2005} and Codebook~\cite{Begel:2010}.

A service company has a large, geographically distributed employee base, with
frequent employee churn at project and organization levels. To maintain
consistent delivery quality, it is essential to manage knowledge at the project
level. Moreover, because services is a highly price-sensitive business, there is
also the need to be cheaper and better, by doing more with fewer or less-skilled
resources. Driven by this need to gain competitive advantage in increasingly
competitive markets, it is essential for companies to build \textit{organization
  memories} that store the collective knowledge of past engagements, processes,
and people to increase productivity and reduce activities that ``reinvent the
wheel.'' Such a system can enable the organization to leverage learnings and
solutions from past services engagements in the context of a new similar
engagement, even when members from the past projects are not around.

The need for organization-level knowledge bases is well established in the
management literature~\cite{davenport2000working,bollinger2001managing}. Equally
well known is the fact that creating an effective organization-wide knowledge
base is very challenging~\cite{McKinsey:1999,Harvard:1999,Ernst:1997}. There are
challenges in: (1) \textit{knowledge creation}---how to codify explicit and
tacit knowledge and motivate individuals to contribute; (2) \textit{knowledge
  retrieval}---data versus information versus knowledge; (3) \textit{knowledge
  governance}---legitimacy, relevance, and quality of contributed
knowledge. Alavi and Leidner~\cite{Alavi:2001} present a good overview of
research issues in organization knowledge management. Prior research suggests
that IT is incapable of capturing organizational
knowledge~\cite{malhotra2004knowledge,mcdermott2000information}, but also
postulates that IT is the strongest enabler for organization knowledge
management systems (OKMS).

Next, we present three scenarios illustrating the need for OKMS in service
companies. Then, we discuss promising research directions based on our
experience with building systems intended to promote knowledge reuse in these
scenarios.

\subsection{Scenarios for OKMS}

In this section, we present three typical scenarios in service delivery that can
benefit from an OKMS.

\subsubsection{Troubleshooting}

One of the common forms of service engagements is application maintenance, where
the expectation from the service provider is to take over a client's
applications and handle service requests for them.  Service requests come in the
form of trouble ``tickets'': users of the applications can raise a ticket,
logging a problem they have experienced. This is similar to defect logging in
bug repositories, such as Bugzilla, where users of an open-source software can
enter the details of a problem that they encountered.  The main difference is
that, in typical software development in open-source communities, there is no
obligation on the development team to address the defects in a timely
fashion. Likewise, even in a product setting, the development team can
prioritize the order in which they address defects.  In services context, the
service provider must resolve the ticket in a timely manner, often under a
service-level agreement. For example, a critical bug must be resolved within
6~hours, failing which there could be financial consequences for the provider.

Software development in service organizations is not pure custom
implementation---often, packaged applications, such as SAP, Oracle, and COTS
products, with client-specific customizations and external libraries are
used. The cause of a problem ticket could be in the customization done for the
client, in the way external code is used, or even a bug in the external code. If
the issue is with the configuration or the external code, it is likely that the
same (or a similar) issue has been resolved previously for another client;
access to that information would make the resolution of the current ticket more
efficient and effective.  Therefore, a knowledge base that stores past resolved
tickets across clients would be a useful organization-wide resource.  In a way,
such a knowledge base would be similar to public question-and-answer forums on
various software languages, tools, and open-source projects on the web.

\subsubsection{Software Development Projects}

Another common form of service engagement is business-process transformation,
where the responsibility of the service provider is to IT-enable a business
process, such as payroll management, vendor management, or order-to-cash, for a
client. Some business processes (\eg payroll management) would be required by
all clients, whereas other processes would be common within a domain only (\eg
claim-management process in the insurance domain).  The client expectation is
that the service provider possesses adequate knowledge of the generic version of
a particular process, creates client-specific variations, and implements the
system. Typically, this requires significant domain expertise.

A knowledge-management system that stores past business process implementations
can help in this scenario. The past solutions need to be organized by domain for
easy retrieval of relevant information. Appropriate documentation that explains
the standard and customized portions of past solutions needs to be available,
along with the solution code. While designing a new business-process solution,
the team can search the repository to learn about variations of the process to
be implemented and, if the client requirements are similar to a past solution,
even reuse the solution in totality or parts. This can reduce the cost and let
the service provider staff the team with people with less domain knowledge.

Code reuse at different levels of granularity---lines-of-code level, API level,
and even complete solutions---has been an area of interest in the
software-engineering community~\cite{Holmes:2013,Reiss:2009}. Much of this work
could be applied in the setting of software services too. However, in the
services context, because code is developed for a particular client, the
intellectual property ownership of the code could become an issue. If the code
is solely owned by the client, making it available in a shared repository would
not be possible. Services organizations often face the challenge of convincing
clients of the benefits of sharing artifacts created as part of a service
engagement into cross-account repositories maintained by the service
providers. Experience reports and user studies that quantify effort reduction
and/or quality improvement from code reuse can help service organizations
convince the clients of the potential benefits of artifact sharing and reuse.

\subsubsection{Service Improvement}

When a client outsources application maintenance to a vendor, one of the key
expectations is that the vendor would bring down their total cost of ownership
over time. This requires the vendor to seek out proactively areas for
improvement in the client application portfolio. One way to do this is via
benchmarking the performance of the client applications against similar
applications in other client landscapes. To illustrate, suppose that a service
company maintains the payroll applications for 10~clients. For nine of the
clients, the monthly ticket volumes range between 5 and 10 tickets, whereas, for
the remaining client (say client~A), the ticket volume ranges from 20 to
50. This indicates that, for client~A, it may be worthwhile to investigate
reasons for high ticket volumes and determine whether preventive actions could
be taken. Moreover, if a similar problem was seen in the past in another
client's payroll system, information about the actions taken for that client
could help the team resolve the problem for client~A.

Many organizations (\eg CAST, Software Engineering Institute) collect
quantitative and qualitative benchmark information for software projects and
business applications based on languages used, number of users, etc.  For a
service company, which does software implementations for multiple clients, an
organization-wide knowledge base that (1) captures key operational metrics per
application (or by problem area) and the past improvement actions taken for each
client, and (2) permits comparisons between clients with similar applications
and problems encountered, would improve upon what is publicly available and be
more useful.

%% There are a number of organizations that gather and report quantitative
%% benchmark information (qualitiative as well as productivity) for software
%% projects and business applications depending on language used to code, number of
%% users etc. CAST, Software Engineering Institute are some examples of such
%% organizations. However, considering services organizations are doing multiple
%% software implementations for multiple clients, an organization-wide knowledge
%% base that (1) captures key operational metrics per application (or at a more
%% granular level, such as by problem area) per client and the past improvement
%% actions taken, and (2) allows comparisons between clients with similar
%% applications and problems encountered, would help augment what is publicly
%% available and be more useful.

\begin{figure}[t]
	\center \includegraphics[width=\columnwidth]{figs/km.png}
        \vspace*{-18pt}
	\caption{Activities involved and tasks performed in building and using
          an OKMS system for a service company.}
        \vspace*{-10pt}
	\label{fig-km}
\end{figure}

\subsection{Research Topics}

Based on such OKMS needs in IBM's service-delivery organization, there have been
efforts toward implementing systems to address the needs. These include a system
for promoting solution reuse in software development engagements centered around
business-process transformation~\cite{Goodwin:2012b}, and a system for sharing
information about problem tickets across client
engagements~\cite{Majumdar:2011}.

Next, we discuss some core research problems that need to be addressed in
creating effective knowledge-management systems in service organizations. At the
simplest level, an OKMS is a database where content can be stored and retrieved
from. However, what content should be stored, how easily can the content be
collected, and the ease with which relevant content can be retrieved determines
the usefulness of an OKMS.  Figure~\ref{fig-km} presents several activities
(shown as block arrows) that need to be performed in creating and using an OKMS;
for each activity, the figure displays relevant topics (shown in the boxes) that
would benefit from further research. The activities are broadly classified into
two categories: knowledge creation and knowledge retrieval.


\subsubsection{Knowledge Creation}

What constitutes useful knowledge in a service company? How can such knowledge
be collected and organized? These are some of the questions that need to be
addressed in knowledge creation. Specifically, we discuss three aspects of
knowledge creation: \textit{crawling}, to collect content from diverse
sources; \textit{parsing}, to translate the content in various formats into a
standard format; and \textit{annotating}, to extract metadata from the content
that can help organize it.

\vskip -5pt
\paragraph*{Crawling} The first activity in building an OKMS is
identifying the data that should be stored in the repository and how to obtain
the data. In general, this activity includes a combination of manually provided
data and automatically crawled data, each of which can have its own peculiar
challenges.

For manual data collection, an organization can require its employees to
contribute learnings and software artifacts to the OKMS. For instance, knowledge
creation can be done via frequently asked questions, where employees outline the
solutions to some common problem tickets they have resolved; alternatively, it
could be done via postmortem reports, usage stories, and experience
reports~\cite{desouza:2005} created by project members after the resolution of a
key project issue. If a client solution involves software development, employees
can be asked to identify reusable components in the software and contribute
them, possibly after generalization, to the OKMS. Manual content creation puts
extra burden on the employees, beyond their normal delivery-related
responsibilities. Thus, an interesting research problem is how to motivate
employees to contribute high-quality content to the
OKMS~\cite{hendriks1999share}. Incentive mechanisms could include ``badges'' as
is done in open-source forums such as stack exchange to encourage question and
answer contributions. In a service company, would reputation-building incentives
be sufficient or would monetary or career-growth incentives be
necessary~\cite{bartol2002encouraging}?

%% An organization has the option of mandating that each of it's employees
%% contributes their learnings to the OKMS system. Some examples of manual
%% knowledge creations are: (1) frequently asked question, where employee outlines
%% the typical solutions to some common problem tickets they have resolved (2)
%% postmortem reports, usage stories, experience reports\cite{desouza:2005}, that
%% project members can create after they solve a key issue in the project. To
%% enable solution reuse, services organizations often invest in creating software
%% product families
%% \cite{clements2002software}. However, here it's pre-anticipated what could be
%% some solutions that could be of interest to multiple clients in a particular
%% domain such as healthcare and then those solutions are created with appropriate
%% points for variability built in, so solution can be customized for any client
%% intending to use it.  Once a project is completed, employees are also encouraged
%% to identify reusable components in software they created and share them in the
%% knowledge base. This approach of manual content creation adds extra burden on
%% the employees. An open research challenge is to motivate employees to contribute
%% high quality content in the knowledge base \cite{hendriks1999share}. Open source
%% forums such as stack exchange have experimented with elaborate incentive
%% mechanisms in forms of badges to ensure questions and answer contributions on
%% their forums \cite{vasilescu2014social}. In services organizations are
%% reputation building incentives enough or incentives should translate to monetory
%% benefits and/or career progression /cite{bartol2002encouraging}?

In addition, or as an alternative, to manual content creation, content can be
collected through automated crawling of the artifacts produced in a project. For
example, complete traceability from requirements through code to test cases,
along with relevant content, can be extracted from Application Lifecycle
Management tools (\eg Rational Team Concert). Similarly, information about
problem tickets and their resolutions can be extracted from ticket-management
systems.  Account teams periodically report important metrics, such as ticket
volumes, code changes, and service-improvement actions taken, on the client
applications being maintained; such reports can be automatically pushed into the
OKMS. The main challenge here is handling the diversity of tools and
technologies used to create software artifacts and related information across
projects.

%% Another approach for content creation is to auto-harvest artifacts produced in
%% the SDLC lifecycle and put these in the knowledge management system. Complete
%% traceability from requirement through code to test cases, along with their
%% content is extractable from Application Lifecycle Management tools and this can
%% act as solution packs to put in the repository. Similarly for ticket resolution,
%% information about problem ticket and it's resolution is extractable from ticket
%% management systems. Account teams periodically report on various important
%% metrics on the applications being maintained in the client landscape such as
%% ticket volumes, code changes, service improvement actions taken. These reports
%% can be auto pushed into the knowledge repository. The main challenge here is
%% handling the diversity of tools and technologies used to create SDLC data across
%% projects.

\paragraph*{Parsing} Once all the requisite data has been pulled from various sources, the main challenge is parsing of the data from various document formats and templates into a standardized format that can be pushed into the knowledge repository. Many of the crawled data is in form of documents in proprietery formats such as Microsoft word, power point slides, visio diagrams, excel sheets. How to automate content extraction from client specific work products into the standard format used by knowledge repository is again a direction for research. One approach is to write model to model transforms where a project admin can specify the mapping \cite{debdoot:2010:scc}. Another challenge is that requisite traceability information that is required to understand complete context of how and why an artifact was produced, might be missing. This is because tools being used to create different SDLC artifacts have not been integrated. There is need to auto infer traceability between artifacts
For example, if making a code commit, the developer puts in a comment like "Fixed Bug \#145", then with high confidence the change is to fix "Bug \#145" and traceability edge between code file and bug report should be auto-created. 
. Tracability inference is an active area of research \cite{spanoudakis2005software}. Some proposed approaches use information retrieval (IR) techniques, others use traceability rules, special integrators, and inference axioms.

\paragraph*{Annotation}

The content put in the OKMS should be organized in such a way that retrieval becomes easier when someone needs it. The general practice is to classify OKMS content against pre-defined taxonomies. One taxonomy is obviously the object type i.e. requirement, code, problem ticket (further segmented into problem description, resolution). Another taxonomy captures the domain the artifact was produced in i.e. industry type and process area e.g. \cite{apqc,bph}. Another taxonomy is the technology used. Organizations spend effort building and maintaining these taxonomies. For every data that is put into the knowledge base, the content needs to be manually categorized against these pre-defined taxonomies. Pattern matching and machine learning based classification approaches /cite{bishop2006pattern} can be used to auto-categorize content to these pre-defined taxonomies. These techniques rely on the availability of equal proportion of positive and negative samples to train a learning model. However, due to unavailability of training data and/or lack of differentiating features, usual learning techniques such as naive bayes, support vector machines, decision trees, might end up not giving desired efficacy (measured as precision and recall). There is need to customize more advanced learning approaches such as adaptive learning, ensemble techniques or develop new techniques that work well with SDLC data. Another interesting area for research is to explore how to help grow the taxonomy over time based on content coming in the knowledge base. Techniques such as clustering \cite{Berkhin06}, topic modeling \cite{Blei:2012} help group together similar looking content and infer topics out of them. 

The content in a OKMS database comes from various clients:. There are strict privacy constraints around what data is client confidential and hence cannot be shared. Within a single artifact itself, there might be small portions of content that are client confidential. E.g. a problem ticket might content the contact information of the user who encountered the information. How to remove client confidential data from artifacts put in the repository, how to anonimize the content so as to not disclose client identity and how to ensure that only authorized users and roles have access.

Once an OKMS system is implemented in an organization, irrespective of the approach to collect content i.e. manual, automated harvesting or hybrid, the repository starts filling up fast. Over a period of one year, the problem ticket repository we setup within IBM collected 750K tickets. Similarly, the business process solution repository has 16000 solutions. However, not all content being put in the repository is high quality and reusable. Hence, it is becomes imperative to be able to filter out useless content. One way to achieve this is by making a human vet every content being pushed in the repository and only content that is deemed high quality is published. Another approach is to ask people who are retrieving and potentially using the content, give feedback on whether they found content useful or not. The third approach that makes for an interesting research direction is to explore how a an automatic quality score can be assiged to each artifact based on content in the artifact, prior reputation of people who authored the content, whether the project was a success or not and so on. In our problem ticket repository, we experimented with calculating a quality score per ticket based on technical versus non-technical content present in the ticket \cite{Majumdar:2011}.


\paragraph*{Summary} To summarize, content creation in OKMS provides multiple oppurtunities where research can contribute. These include: what and how to extract content from SDLC repositories and proprietry document formats, how to infer traceability between different artifacts, how to classify and categorize content, how to maintain privacy, estimate quality and motivate employees to contribute high quality content. 



\subsubsection{Knowledge Retrieval}

Challenges in knowledge retrieval are related to the search capabilities offered
by the OKMS and the effort required from users in determining the usefulness of
the recommended data. We discuss the following aspects of knowledge retrieval:
indexing and searching, search result organization, and advanced analytics.

\paragraph*{Indexing and Searching} The simplest approach for searching
is \textit{keyword-based search}, in which users specify the words they are
looking for and the system returns all artifacts that contain the specified
words. More sophisticated systems use \textit{faceted search}, where the user
can navigate a hierarchical structure (\eg a taxonomy) and select values from
predefined categories.  Despite the availability of many search technologies,
studies (\eg \cite{idc,idc2}) have shown that retrieval of relevant information
from organizational repositories remains challenging, with users being
successful in less than 50\% of their attempts in searching for information.

Language-based information-retrieval techniques~\cite{manning2008introduction},
on which many knowledge systems are based, may be inadequate in dealing with the
types of repositories we are talking about---that contain not just a collection
of artifacts, but a network of linked artifacts.  Graph databases for storage
and semantic search techniques~\cite{Guha:2003} or extending keyword search for
graphs~\cite{kacholia2005bidirectional} may be more promising in this scenario,
and are worthy of research investigation.

%% \paragraph*{Index and Search} Indexing is how the knowledge repository stores
%% data internally. Search features then work on this index.  Typical ways to
%% retrieve content from a knowledge base are: (1) keyword based search where user
%% specifies a couple of words (s)he is looking for and all artifacts in the
%% repository that contain the words from user query are returned, (2) faceted (or
%% navigational) search where user is shown a hierarchy structure (taxonomy) and
%% can browse information by choosing one or more values from each of the
%% pre-defined categories. Various language based information retrieval
%% models \cite{manning2008introduction} such as vector space models, probablistic
%% models, latent semantic index exist that can be used here. But inspite of easy
%% to use search technologies being available, prior studies report that knowledge
%% retrieval from organization wide repositories remains a challenge. As
%% per \cite{idc,idc2}, while employees spend 15\% to 35\% of their time searching
%% for information in an enterprise, they are successful less than 50\% of the time
%% in finding what they are looking for. Most existing OKMS systems use language
%% based IR models to store and retrieve knowledge. However, as we saw in content
%% creation section, the repository is not just a collection of artifacts but a
%% network of linked artifacts. Use of graph databases for data storage and
%% retrieval techniques such as semantic search techniques \cite{Guha:2003} or
%% optimizing keyword search for graphs \cite{kacholia2005bidirectional}, are worth
%% exploring.

An important factor in performing accurate search is the expressiveness of query
construction. Simple keyword queries, consisting of a set of words, may not be
discriminative enough to return accurate results. A few approaches have been
developed to address this problem, for example, by giving more weights to words
that appear in the title of a problem ticket than words that appear in the
ticket description~\cite{Sinha:2012}, and using separate queries for different
types of information, such as description, application information, and stack
trace, in a problem ticket~\cite{Ashok:2009}.

Another question pertains to assigning weights to clauses in the query while
ranking the search results~\cite{Debdoot:2011:bpm}. Moreover, the idea
of \textit{contextual search}~\cite{wen2004probabilistic,kraft2005q}, which
attempts to capture the user's information needs better by augmenting the query
with contextual information extracted from search context, has shown promising
results. Further research along these directions---in the context of the
knowledge needs in service delivery---investigating different ways of creating
contexts, using contexts in constructing queries, and assigning weights to query
clauses would be interesting.

%% Non-availability of content or poorly organized content can be one reason for
%% this. Another reason could be the inadequacy of the query itself that are used
%% to retrive the content. Suppose a user is trying to find problem tickets that
%% resolved similar issues to what (s)he has been assigned. (S)he would pick up a
%% couple of words from the ticket that describe the problem and use it to query
%% the knowledge base. However, these words might not be discriminative enough and
%% user might end up getting too many or too little search hits. Another approach
%% could be to use complete content in the ticket and use it as query. However in
%% this case, the search engine might end up returning irrelevant results as equal
%% weightage was given to all words in the problem ticket. \cite{Sinha:2012} tries
%% to address this issue by giving more weightage to those words in the query that
%% belong to ticket title. \cite{Ashok:2009} parses out different datatypes from a
%% problem ticket such as description, application information, stack trace and
%% using a different query/search mechanism for each. E.g. instead of just
%% specifying "null pointer exception", the query generated from a problem ticket
%% could be---description: contains following bag of words \{null, pointer,
%% exception\}, process: is equal to "order to cash", stack trace: contains
%% foo.*bar. The results obtained from such a query are likely to be more precise
%% than what a keyword search would yield. A challenge with this approach is how to
%% compose the various clauses in the query---are they "anded" or "ored" or a
%% combination. Another challenge is how much weightage to give to each clause when
%% ranking the search results \cite{Debdoot:2011:bpm}. Moreover, recently
%% contextual search \cite{wen2004probabilistic,kraft2005q} has been gaining
%% momentum. Contextual search tries to better capture a user's information need by
%% augmenting the query with contextual information extracted from search
%% context. It would be an interesting direction for future research to see for
%% each of the knowledge needs in services organizations, what can make up the
%% context, how to use this context to create the query and how to weigh different
%% clauses in the query.

\paragraph*{Search Result Organization}
Although more powerful querying capabilities can help improve the accuracy of
search results, a search-based system would in general return multiple
results. The next question then is: how much effort does it take for the user to
sift through the results to find relevant information? Simple approaches such as
highlighting matching keywords will not work for complex queries; more
sophisticated techniques that help the user easily comprehend the relevance of
the results are needed. One such approach is \textit{summarization}, which has
been used in text-processing domains to end users get a quick overview of
information; summarization has also been applied to bug
reports~\cite{Mani:2012,Rastkar:2010}. Extending this notion to other types of
artifacts, such as code, requirements, and solution designs, is an open topic
for research.

Another types of analysis that can improve the search results is detection and
grouping of duplicates or near-duplicates. By grouping together such artifacts
and highlighting the variances in seemingly similar artifacts, the system can
reduce information overload on the user.  Detection of duplicate bug reports has
been widely researched
(\eg \cite{wang2008approach,sun2010discriminative}). Developing such analyses
for other types of artifacts, from a services OKMS perspective, would be useful.

%% \paragraph*{Search Result Organization}
%% Any search based system would return multiple results. For the user it becomes a
%% chore in itself to go over each of the search results and identify if it is
%% relevant or not. Techniques that can help the user easily comprehend the
%% relevance of the returned results are much needed. Text based search engines
%% highlight the matching words between query and content of the artifact
%% returned. However, when the query becomes complex (as above), keyword
%% highlighting is not of much help. In text processing domain, summarization is
%% one approach that is used to help end users get a quick idea of what the content
%% is about. In \cite{Mani:2012,Rastkar:2010} authors present different techniques
%% to summarize bug reports. Research can help in developing summarization
%% techniques that work for other SDLC artifact such as code, requirements,
%% solution design documents and so on. Further, out of the search results returned
%% many artifacts could be duplicates or near duplicates of each other. In order to
%% reduce information overload for end user, the OKMS system should be able to
%% group together duplicates and near duplicates and be able to highlight the
%% variances in seemingly similar artifacts returned in the search
%% results. Duplicate bug report
%% detection \cite{wang2008approach,sun2010discriminative} has been a widely
%% researched topic. Similarly for each of the artifact of interest from a services
%% OKMS perspective, a customized duplicate detection approach might be needed.

\paragraph*{Advanced Analytics}
Most knowledge-management systems today stop at providing search
capabilities. Given the magnitude of data that can be collected in OKMS, there
is the opportunity of implementing advanced analytics that go beyond by
supporting decision making. Consider the scenario of troubleshooting where, to
resolve a ticket, the user searches through past resolved tickets to find
similar tickets.  Based on the past tickets and the current context, the user
has to formulate potential hypotheses about the root cause, decide which
hypotheses are applicable, and then investigate the solutions. The OKMS would be
much more useful if it could automatically generate potential hypotheses about
root causes and solutions for the user to investigate. Such analytics could also
be developed for other scenarios such as service improvement.

%% \paragraph*{Advanced Analytics}
%% OKMS today stops at providing search capabilities. Given a user query, the
%% system returns back matching artifacts but makes no attempt at making any
%% deductions. Consider the scenario of troubleshooting. A user is searching
%% through the past resolved ticket repository to see if similar issues were
%% resolved. There are multiple reasons why the issues could have arisen. Based on
%% past tickets, the user needs to come up with potential hypothesis of why the
%% issue could be arising and based on conditions (s)he is seeing in the current
%% landscape decide what hypothesis is applicable and then pick the appropriate
%% solution. From such a user's perspective the OKMS system would be more effective
%% if it were able to auto-generate these potential root causes and solutions
%% hypothesis. Consider the service improvement use case, the accout team needs
%% support from OKMS to identify similar projects, then compare the problems seen
%% in the current project with different kind of issues arising in similar
%% projects, then judge whether current team is doing better or worse and then
%% decide to take some action. What kind of capabilities are needed in OKMS system
%% to be useful in above scenarios is another area for research.

In general, there are two aspects of organization knowledge management:
codification and personalization~\cite{hansen2000s}. Our discussion of OKMS has
so far focused on the codification aspect, where knowledge is carefully captured
and stored in a knowledge base for use by anyone in the company. The
personalization aspect ties knowledge to specific people; the main purpose of
knowledge management here is to help the user find and connect with people who
have the requisite knowledge, not store the knowledge. This strategy is
especially appropriate in cases where knowledge cannot be easily codified, such
as the knowledge for handling situations that require complex decision
making. Expertise browsing, recommendation, and mining have been explored in
various contexts (\eg \cite{Balog:2006, McDonald:2000, Mockus:2002}). Extending
these ideas to the context of service delivery, where the OKMS contains data
from multiple projects in different contexts, is an area where further research
can help.

%% There are two techniques for organization knowledge management, codification or
%% personalization \cite{hansen2000s}.  Till now what we have discussed is what is
%% called codification. Here knowledge is carefully captured and stored in the
%% database, where it can be accessed and used by anyone in the company. This
%% strategy allows many people to search for and retrieve knowledge without having
%% to contact the person who originally developed it. This opens up the possibility
%% of achieving required scale in knowledge reuse in services
%% organizations. Another strategy for knowledge management is
%% personalization. Here knowledge is tied to person who developed is and is shared
%% mainly through direct person-to-person contacts. The chief purpose of knowledge
%% management system here is to help people find and connect to other people who
%% have requisite knowledge, not to store it. This strategy works well when
%% knowledge cannot be codified especially knowledge required to handle situations
%% that require complex decision making. Expertise browser \cite{Mockus:2002},
%% expertise recommender \cite{McDonald:2000} have attempted to identify experts on
%% various topics within a project. \cite{Balog:2006} has explored expertise mining
%% and recommendation in predominantly document based OKMS systems. What
%% information to use to mine expertise when repository contains SDLC data from
%% multiple projects, how to match your current context and job profile to suggest
%% people you should have in your network is another scenario where research can
%% help.

%% \paragraph*{Summary} To summarize, content retrieval in OKMS system provides
%% open research problems in query generation and use of context to augment
%% queries, search result summarization, duplicate detection, use of semantic
%% search to improve search relevance. Further the retrieval capabilities need to
%% move beyond just search to provide capabilities such as question-answer
%% generation, cross-account comparison/benchmarking and expertise recommendation.


