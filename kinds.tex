\section{Services Engagements: A Closer Look}

At the highest level, software services are of three kinds: consulting, systems
integration, and application services. \textit{Consulting} refers to
business-level consulting, in which domain experts work with a client to advise
the client on major decisions and high-level design of a solution.
\textit{Systems integration} refers to assembling a solution from off-the-shelf
software and hardware components, with relatively little custom-programming
component.  We focus only on \textit{application services}---this is the kind
that relates most closely to software engineering.  To make the setting
concrete, we illustrate a service engagement that focuses on application
services.

Consider a large health-care management company that needs to make sure that
their IT systems are compliant with a newly passed patient-privacy
legislation. Over the years, this company has accumulated a portfolio of a dozen
applications implemented in a variety of technologies. It did not make business
sense to keep in-house staff around to be able to make transformational changes
to the application. The company decides to use a software services company to
make this change. While they are at it, they decided to outsource support
activities to the services company, to further reduce in-house IT costs.
Vendors (\ie services companies) are invited to submit proposals to transform
the application suite, test it, and provide steady-state maintenance service.

\begin{figure}[t]
\centering
\includegraphics[width=0.85\columnwidth, clip, trim = 0mm 50mm 0mm
  10mm]{figs/phases.pdf}
\vspace*{-10pt}
\caption{The phases in a typical software services engagement.}
\vspace*{-15pt}
\label{fig:phases}
\end{figure}

Software services vendors respond to the request for proposals with a plan of
what they would do, along with the pricing. To differentiate themselves, they
need to show the innovations that they bring to the table that help the
health-care management company in terms of cost or quality. For example, a
differentiator could be the fact that the vendor has previously worked on similar
transformational effort for another health-case management company, and was able
to demonstrate 20\% reduction in IT support costs. 
Vendors also offer service-level agreements, so that the
health-care management company can hedge their risk.  Typically, service-level
agreements include financial charges if certain milestones are not met, or if
support quality does not measure up to benchmarks. The vendors also need to
perform \textit{due diligence}, in which they assess their own risk in taking on
the application portfolio from the health-care management company: the messier
the portfolio, the more it would cost to transform and support it.

The winning vendor would then go through phases of transitioning,
transformation, and steady-state maintenance; see
Figure~\ref{fig:phases}. \textit{Transitioning} involves taking over the
application portfolio from its owner. Some amount of human-level knowledge
transfer from the client to the vendor takes place by way of workshops, but
typically, a lot is left for the vendor to discover from the available
artifacts. In the \textit{transformation phase}, the client and the vendor
collaborate to work out detailed requirements, and the vendor allocates human
resources to carry out implementation and testing. The implementation and
testing phases resemble a development or a legacy transformation project at a
product company.  Finally, the transformed portfolio enters \textit{steady-state
  maintenance phase}, where field bugs are handled and customer support is
provided.  The latter phases are labor intensive, and the vendor has to organize
the flow of work through a large workforce.

The vendor relies heavily on \textit{knowledge management} through all these
phases: the project must go on with as little as possible reliance on particular
individuals who have an intimate knowledge of the application portfolio. Not
only are the client's people unavailable, there is significant churn in the
vendor's staff as well.  Additionally, a governance and risk management plan is
put in effect for the entire lifecycle of the engagement, so both the client and
the vendor are in sync and the continuation of the engagement makes business
sense.

Services engagements vary widely. A company could use multiple vendors to carry
out the work---for example, one vendor for implementation, another for testing,
and yet another for steady-state maintenance.  Moreover, although this example
was in the context of an existing application portfolio, sometimes services
engagements start from the design and build phase, also known as ``green field''
projects.

%Consulting vs. Systems integration vs. Application Services.

%Design and build.
%
%Enhancements.
%
%Maintenance.
%
%Talk about organization of vendors. By service lines? By industry? By technologies? Look at Steve Kagan's white paper.

