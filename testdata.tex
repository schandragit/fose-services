%\subsection{Test Data Generation}

Enterprise applications are typically database-oriented, in that they make queries over databases and the program flow depends on the data retrieved as a result of a query.  In order to test whether the application satisfies its requirements, sufficient amount of test data must be made available in the database. Consider, for example, a banking application, in which one of the requirement to exercise is that if a customer makes a certain fixed deposit, and the person is a senior citizen, and also holds a checking account in the bank, then an addition 0.25\% is paid as interest. To test that this situation is handled correctly in the application, there should exist a customer in the database such that his or her age qualifies the person as senior citizen and the person also has a checking account. Notice that typically, customer data and account data would be maintained in separate database tables. Typically business requirements are complex, and there need to be specific entries in multiple tables for a scenario to work out.

When a client gives a testing contract, the client may expect the service provider to exercise all the business rules, but may not give enough sample data in the database to be able to exercise these rules.  It then becomes the responsibility of the service provider to figure out the requisite test data to be present in the pertinent tables.

Randomly generated test data  cannot be expected to suffice for enterprise applications with complex rules. Also, systematic test generation approaches based on program analysis cannot be expected to tackle enterprise application, which use a mix of multiple language and database technologies in their implementation. The database community has done some work in creating test data for SQL queries, and that may be relevant; however, as mentioned before, enterprise applications are implemented with a mix of technologies.

We see a big opportunity for research contributions in this subarea. How to take a specification of the application as testing criteria, and automatically populate database that would enable various scenarios that fulfill the test criteria to be exercised.
