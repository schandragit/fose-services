
\documentclass{sig-alternate}

\usepackage{amssymb}
\usepackage{verbatim}
\usepackage{times}
\usepackage{graphicx}
\usepackage{url}
\usepackage{xspace}
\usepackage{float}
\usepackage{latexsym}
\usepackage{natbib}
\usepackage{alltt}
\usepackage{color}
\usepackage{textcomp}
\usepackage{balance}

\bibpunct{[}{]}{,}{n}{}{}

\newcommand{\ie}{\textit{i.e.,} }
\newcommand{\eg}{\textit{e.g.,} }

\def\denseitems{
  \itemsep1pt plus1pt minus1pt
  \parsep0pt plus0pt
  \parskip0pt\topsep0pt}

\begin{document}

\conferenceinfo{FOSE}{'14, May 31 -- June 7, 2014, Hyderabad, India}
\CopyrightYear{2014}
\crdata{978-1-4503-2865-4/14/05}

\title{Software Services: A Research Roadmap}

\numberofauthors{4}

\author{
\alignauthor Satish Chandra\titlenote{\small This author was formerly at IBM.}\\
       \affaddr{Samsung Electronics}\\
       \affaddr{San Jose, CA}\\
       \email{schandra@acm.org}
\alignauthor Vibha S Sinha\\
	\affaddr{IBM Research}\\
	\affaddr{New Delhi, India}\\
	\email{vibha.sinha@in.ibm.com}
\and
\alignauthor Saurabh Sinha\\
	\affaddr{IBM Research}\\
	\affaddr{Bangalore, India}\\
	\email{saurabhsinha@in.ibm.com}
\alignauthor Krishna Ratakonda\\
	\affaddr{IBM T J Watson Research}\\
	\affaddr{Yorktown Heights, NY}\\
	\email{ratakond@us.ibm.com}
}

\maketitle

\begin{abstract}

Software services companies offer software development, testing and maintenance
as a ``service'' to other organizations.  As a thriving industry in its own
right, software services offers certain unique research problems as well as
different takes on research problems typically considered in software
engineering research. In this paper, we highlight some of these research
problems, drawing heavily upon our involvement with IBM Global Business Services
organization over the past several years.  We focus on four selected topics: how
to organize people and the flow of work through people, how to manage knowledge
at an organizational level, how to estimate and manage risk in a services
engagement, and finally, testing services. These topics by no means cover all
areas pertinent to software services; rather, they reflect ones in which we have
personal perspectives to offer.  We also share our experience in deployment of
research innovations in a large service delivery organization.

\end{abstract}

\category{D.2.9}{Software Engineering}{Management}
\category{D.2.5}{Software Engineering}{Testing and Debugging}

\terms{Management, Measurement}

\keywords{Software services, distributed development, testing, knowledge management, risk management}

\section{Introduction}
% test change
\label{sec:intro}

\section{Kinds of Services Engagements}

\section{Organizing People and Work}
\label{sec:global}

Globalization and the implicit challenges of distributed development have
become the norm in recent years for large enterprises~\cite{glo24,glo26}.
Evaristo~\cite{glo27} categorizes the nature of these challenges into five
critical areas: perceived distance (geographical and temporal), national
culture (languages, accepted work patterns), development methodology
(similarity in processes), task structure (clarity and structure to team
hand-offs) and organizational distance. The premise is that as the distance
between the teams along these critical dimensions increases, the overhead and
difficulties associated with distributed development become more
prominent. Collaborative development platforms that promote structured
interaction between team members can help make distributed development more
efficient~\cite{glo28,glo29}. Sharing expertise and the ability to leverage
best practices across engagements is a key part of ensuring efficient
collaboration -- Expertise Browser~\cite{glo30} and Hipikat~\cite{glo31} are
examples of tools that enable sharing at the level of processes and
artifacts. Given the rapid churn of resources in the global workforce, a push
model of delivering information in situ to developers is important and many of
the proposed tools~\cite{glo29,glo31} support this model. The adoption of
Cloud, API centric approach to development and agile methods are also helping
to reduce the inefficiencies inherent in distributed development. Tooling that
can help efficiently leverage a global workforce remains an active and
important area of research.

The existing literature on distributed development has mostly focused on the
need of one organization engaged in carrying out work globally for reasons
of efficiency and cost.  Services companies have additional concerns.  They
serve multiple large clients, and the work of each of those clients has to
be carried out in the distributed fashion.  In the earlier days, services
companies grew essentially distinct teams to serve distinct clients.  This
turns out to be not optimimal from a resource utilization point of view.

Recent trend has been towards \textit{competency centers}, which are virtual
entities that specialize in a specific software skill set.  The advantages of
competency center approach are resource multiplexing as well as specialization.
For example, a testing competency center could serve multiple clients, and
could be the home for testing expertise within the company. 

IBM has been a pioneer in
this field---their approach to structured delivery of IT services by a globally
distributed delivery team is called Application Assembly
Optimization~\cite{gloaao}. Other major IT services companies have adopted
variations of this approach.

\textit{Work envelopes} are the central construct in the competency-center approach for
structured delivery of software projects that enable disparate competency
centers to come together on-demand, in the context of a client engagement.  Work
envelopes represent a standard communication vehicle between distributed teams
by which each work order is authored, transported, and delivered. Work envelopes
include workflow, instruction (normative guidance), metrics collection, and risk
management/exception handling mechanisms. Each work envelope constitutes a
subset of activities that are bound to a larger project-wide work-breakdown
structure managed through a traditional project-management tool. 



%\textbf{Prior Material}
%
%The last few years have seen an accelerated trend toward organizations utilizing
%a globally integrated development model for complex IT solution
%implementation. In the first wave of the move toward globally integrated
%development, the emphasis was on reducing cost through labor arbitrage.  The
%competition among the major industry players generated the need for aggressive
%differentiation beyond lower cost---improving quality and reducing
%time-to-market also became important factors. However, traditional project
%management techniques that work well with small co-located teams do not scale
%well to a global workforce.  To be effective, the development process should be
%supplemented by an over-arching procedural and architectural framework that
%implicitly partitions the development process for effective globally sourced
%development. Formal process modeling, strict enforcement of process controls,
%and monitoring the software development process have long been touted as the
%right way to streamline global delivery~\cite{glo32}.
%
%\begin{figure}[t]
%\centering
%\includegraphics[bb= 28 0 740 355, scale=0.32]{figs/glocomp.jpg}
%\vspace*{-13pt}
%\caption{The globally integrated development model with competency centers.}
%\vspace*{-16pt}
%\label{glofig1}
%\end{figure}
%
%As large enterprises adapted traditional software delivery models to distributed
%development, the natural tendency has been to establish competency centers that
%cater to specific software-skill needs. The adoption of a disciplined model for
%delivery has resulted in new organizational efficiency, reduced process
%variance, and improved execution discipline. Consider the delivery of a large IT
%project that involves distributed teams with hundreds of people working on tight
%schedules to deliver mission-critical software. Many such projects fail to
%deliver on time and within budget because of the significant overhead involved
%in discovering the roles and responsibilities of teams, establishing
%communication protocols, and determining how software will be assembled, tested,
%and deployed at the client site. Competency centers that standardize key stages
%of the delivery process
%%% for customization of software from major vendors
%can reduce this overhead significantly.
%%% for a majority of enterprise-class IT projects.
%
%Figure~\ref{glofig1} shows the vision of the globally integrated enterprise by
%focusing on one of the key pain points that inhibit this vision---a structured
%way of distributing work to remote teams that promotes collaboration and
%enforces good governance practices.  Some of the key organizational entities in
%this approach include governance and operations teams, a design center that
%builds the architecture in support of client requirements, and the
%\textit{competency centers} that deliver the work. IBM has been a pioneer in
%this field---their approach to structured delivery of IT services by a globally
%distributed delivery team is called Application Assembly
%Optimization~\cite{gloaao}. Other major IT services companies have adopted
%variations of this approach.
%
%Work envelopes are central constructs in the competency-center approach for
%structured delivery of software projects that enable disparate competency
%centers to come together on-demand, in the context of a client engagement.  Work
%envelopes represent a standard communication vehicle between distributed teams
%by which each work order is authored, transported, and delivered. Work envelopes
%include workflow, instruction (normative guidance), metrics collection, and risk
%management/exception handling mechanisms. Each work envelope constitutes a
%subset of activities that are bound to a larger project-wide work-breakdown
%structure managed through a traditional project-management tool.
%
%\textbf{End of Prior material}

\subsection{Research Topic: Efficient Competency Centers}

Although the competency-center model has now become central to many IT vendors
and large enterprises, it has seen less interest in literature as the focus is
on large-scale development with a global workforce, which is hard to replicate
in an academic setting. However, there are several accessible research problems
that can significantly improve the performance of these competency centers. We
highlight five such problem areas below

\begin{enumerate}

\item Work envelopes: Central to the concept of a competency center is the
  ability to partition software projects into granular work envelopes. For
  certain classes of development and maintenance activities---for example, in
  the area of packaged applications---partitioning is natural because tasks are
  repeatable and the skill requirements for a class of tasks is easy to
  establish. Is there an equivalent construct for more complex software
  development?

\item Measurement: In a large competency center, performance measurement is key
  to ensuring efficient operation. In some ways, a competency center, due to the
  repeatable and comparable nature of tasks it undertakes, makes performance
  measurement easier. However, we find that the information reported by the
  practitioners of a competency center is often unreliable due to the complex
  structure of incentives involved. Can we leverage the structure of a
  competency center to automate the collection of information, thereby improving
  its reliability? Can we create checks and balances in information collection
  to validate performance measurement?

\item Estimation: Competency centers deal with large volumes of repeatable tasks
  belonging to a few categories that are performed by dedicated teams. We
  observed that this typically reduces the variance in the estimated time
  required to perform these tasks. The reduction in variance is a consequence of
  the predicability of the inputs and outputs of the tasks, standardized
  process, and a community of developers who can share experiences and
  know-how. In turn, this improves our ability significantly to predict the
  effort needed to complete complex software projects.

\item Governance: Enterprises have extended traditional project management
  techniques to include competency centers. However, it is unclear whether a
  project-centric governance model provides the right set of checks and balances
  to manage efficiently the output of a distributed workforce that is working on
  several projects simultaneously.

\item Planning and scheduling: Traditional calendar-day project planning is
  highly inefficient in the context of a competency center because development
  tasks can finish ahead of or behind schedule. Because a practitioner is not
  tied to a particular project in a competency center, calendar-day planning
  requires constant replanning to ensure that we are fully utilizing available
  capacity. A better approach would be to use a queue-based model, where the
  queue is managed centrally to re-prioritize the tasks in a practitioner's
  queue to ensure on-time delivery. From the practitioner's perspective, they
  move to the next task in the queue after completing the task at hand---if
  they fall behind schedule, the remaining tasks in their queue can be moved to
  other practitioners.

\end{enumerate}

\subsection{Toward Distributed Marketplaces}

The logical evolution of the competency center model that we discussed in the
last section is to extend beyond traditional organizational boundaries---in
essence, towards a competitive distributed services marketplace. Instances of
such marketplaces are already appearing; for example, services that provide
coding expertise for hire, such as RentACoder ({\small
  \url{www.rentacoder.com}}) and TopCoder ({\small \url{www.topcoder.com}}) are
simple examples of this model. Currently, these are seen as a novelty and, thus,
have gained little adoption in large enterprises. In the rest of this section,
we explore what will it take to make this mainstream.

A services delivery marketplace has three key players: provider, consumer, and
marketplace enabler. Consumers can use the enabler to ascertain the risks that
in sourcing from a particular provider. Providers in turn get the ability to
price their services competitively based on delivery record. The enabler
provides core enabling services, such as decomposing large or complex projects
so they can be executed in parallel by different providers, managing complex
projects executed in parallel by different providers, providing service-level
guarantees, and performing service request validation. It is important to note
that our marketplace concept is an abstraction for efficient service
delivery. Thus, all three players can belong to the same organization, a
different organization (in the true global-marketplace sense), or any
combination thereof.

Traditional IT vendors still play a role in this conceptual services
marketplace; however, small or marginal players also have a chance to compete
and grow their reputation. In this setup, there are incentives for every one to
participate. For traditional IT vendors, it helps offload low margin IT services
to smaller vendors with less overhead. For customers, it provides an opportunity
to outsource small work without entering into expensive long-term contracts. For
the marketplace providers---a role we think will be played by traditional IT
vendors---it is a chance to profit from helping both providers and consumers
benefit from the marketplace by offering some guarantees.

In contrast to a marketplace for complete products, service delivery is
inherently more challenging as it is difficult to capture the intent of the
consumer when a service request is created. The service request should not only
set forth the details of the work to be performed, but also specify risk
mitigation, change management, periodic reviews, documentation requirements, and
the exit criteria in a clear fashion to set the right expectations for both the
consumer and the provider.  For example, if a code development project being
worked on by a provider is not progressing as per the agreed schedule, a
risk-mitigation activity may even involve canceling the service agreement. If
such risk-mitigation activities are clearly spelled out during service-request
creation, it may open up avenues for misunderstanding. We anticipate that the
service requests in such a marketplace will not be full-fledged software
projects, but of the size that can be performed in a few days or weeks by a
small team of developers or even individuals. They could have the elements of an
agile or waterfall development process. Figure~\ref{glomarketplace} shows the
conceptual instantiation of such a marketplace.

To make this concept practical, there are many challenges to be overcome, some
of which are common to the competency center model. The primary challenge is
reducing the overhead required to partition and distribute a large software
project into independent units of work that can be done by skilled developers
with little or no reference to the overall project context. Open-source software
development is a case in point where complex software has been developed
successfully through a very loosely organized set of developers. For large
enterprises, this mode of development may not be suitable because it does not
offer appropriate security, governance, and schedule controls. Can we find a
balance between the need to monitor and manage the outcome with the agility that
a truly distributed services marketplace can bring to the table?

\begin{figure}[t]
\centering
\includegraphics[bb=13 20 530 365, scale=0.53]{figs/glomarketplace.jpg}
\vspace*{-15pt}
\caption{Conceptual illustration of a distributed services marketplace.}
\vspace*{-15pt}
\label{glomarketplace}
\end{figure}

Theoretical analysis can point to optimal regimes and controls for the efficient
use of a services marketplace. For example, Ranade and
Varshney~\cite{glo-ranade} use game-theoretic models to offer insights into some
key questions about crowd-sourcing in general: what type of tasks should be
crowd-sourced and under what circumstances? Their conclusion is that the types
of tasks (specialized or generic) and the distribution of skills in the
available pool of people has a big impact on what can be effectively outsourced.






\section{Knowledge Management}
\label{sec:km}

Software development is inherently a knowledge-intensive activity. Software
designers and developers leverage their software-development skills, along with
domain knowledge, past experiences, and the knowledge of team members, to solve
the problem at hand, such as implementing a new feature or resolving a bug.  In
small, collocated teams, knowledge management is not a big challenge---people's
expertise on different parts of a system is typically known. New team members
can use informal communication channels to identify experts and seek their help
as needed.

However, as teams increase in size or become geographically distributed,
knowledge management starts to become challenging. In large or distributed
teams, system knowledge---\eg expertise, dependencies, best practices---is
spread across multiple people, locations, and (in the case of outsourcing) even
organizations~\cite{Desouza:2006}. In such projects, a knowledge-management
system is needed to create \textit{project memories} that can serve different
needs. First, it should assist new team members in understanding the project
with little or no face-to-face guidance and identifying experts to reach out to
for questions. Second, the system should help existing team members identify the
artifacts relevant, and the people they might need to coordinate with, for
performing a task. Existing attempts at creating such project memories include
Hipikat~\cite{Murphy:2005} and Codebook~\cite{Begel:2010}.

A service company has a large, geographically distributed employee base, with
frequent employee churn at project and organization levels. To maintain
consistent delivery quality, it is essential to manage knowledge at the project
level. Moreover, because services is a price-sensitive business, there is also
the need to be cheaper and better, by doing more with fewer or less-skilled
resources. Driven by this need to gain competitive advantage in increasingly
competitive markets, it becomes essential for companies to build
\textit{organization memories} that store the collective knowledge of past
engagements, processes, and people to increase productivity and reduce
activities that ``reinvent the wheel.'' The intent behind such a system is to
ensure that the knowledge of past or current employees is available to other
employees when the need arises. Thus, the organization should be able to
leverage learning and solutions from past client engagements in the context of a
new similar engagement, even when members from the past projects are not around.

The need for organization-level knowledge bases is well established in the
management literature~\cite{davenport2000working,bollinger2001managing}. Equally
well known is the fact that creating an effective organization-wide knowledge
base is very challenging~\cite{McKinsey:1999,Harvard:1999,Ernst:1997}. There are
challenges in: (1) \textit{knowledge creation}---how to codify explicit and
tacit knowledge and motivate individuals to contribute; (2) \textit{knowledge
  retrieval}---data versus information versus knowledge; (3) \textit{knowledge
  governance}---legitimacy, relevance, and quality of contributed
knowledge. Alavi and Leidner~\cite{Alavi:2001} present a good overview of
research issues in organization knowledge management. Prior research suggests
that IT is incapable of capturing organizational knowledge
\cite{malhotra2004knowledge,mcdermott2000information}, but also postulates that
IT is the strongest enabler for organization knowledge management systems
(OKMS).

Next, we present three scenarios illustrating the need for OKMS in service
companies. Then, we discuss promising research directions based on our
experience with building systems intended to promote knowledge reuse in these
scenarios.

\subsection{Scenarios for OKMS}

In this section, we present three typical scenarios in service delivery that can
benefit from an organization knowledge management system.

\subsubsection{Troubleshooting}

One of the common forms of service engagements is application maintenance, where
the expectation from the service provider is to take over a client's custom
application and handle service requests for it.  Service requests come in the
form of trouble ``tickets'': users of the applications can raise a ticket,
logging a problem they have experienced. This is similar to defect logging in
bug repositories, such as Bugzilla, where users of an open-source software can
enter the details of a problem that they encountered.  The main difference is
that, in typical software development in open-source communities, there is no
obligation on the development team to address the defects in a timely
fashion. Likewise, even in a product setting, the development team can
prioritize which defects they are going to address first.  In service context,
the service provider is supposed to resolve the ticket in a timely manner, often
under a service-level agreement. For example, a critical bug must be resolved
within 6~hours, at risk of financial consequences for the provider.

Software development in service organizations is not pure custom
implementations. In many cases, packaged applications, such as SAP, Oracle, and
COTS products, with client-specific customizations and external libraries are
used. The cause of a problem ticket could be in the customization done for the
client, in the way external code is used, or even a bug in the external code. If
the issue is with the configuration or the external code, it is very likely that
the same (or a similar) issue has been resolved previously in the context of
another client. Thus, if the person attempting to resolve the current ticket had
access to previously resolved tickets addressing similar problems, they could be
much more efficient and effective at their task.  Therefore, a knowledge base
that stores past resolved tickets across clients would be a useful
organization-wide resource. 

In a way, such a knowledge base would be similar to public question-and-answer
forums we see on various software languages, tools, and open-source projects on
the web.

\subsubsection{Design-build Projects}

Another common form of service engagement is business-process transformation,
where the expectation from the service provider is to IT-enable a business
process, such as payroll management, vendor management, order-to-cash, for a
client. Some business processes (\eg payroll management) would be required by
all clients, whereas other processes would be common in a particular domain (\eg
claim-management process in the insurance domain).  The client expectation is
that the service provider possesses adequate knowledge of the generic version of
a particular process, creates client-specific variations, and implements the
system. Typically, this requires that the vendor team working on the project has
significant domain experience.

A knowledge-management system that stores past business process implementations
across the organization can help in this scenario. The past solutions need to be
organized by domain to make retrieval of relevant information
easier. Appropriate documentation that explains the standard and customized
portions of past solutions needs to be available, along with the solution
code. While designing a new business-process solution, the team can search
through the repository to learn about the variations of the process to be
implemented and, if the client requirements are not too different from a past
solution, even reuse the solution in totality or parts. This can reduce the
overall cost and also potentially let the service provider staff the team with
people with less domain experience. 

Code reuse, at different levels of granularity---lines-of-code level, API level,
and even complete solutions---has been an area of interest in the
software-engineering community~\cite{Reiss:2009,Holmes:2013}. Much of this work
could be applied in the setting of a service company too.

\subsubsection{Service Improvement}

When a client outsources its application maintanence to another organization,
one of the key expectations is that the vendor would bring down their total cost
of ownership over time. This requires the vendor to proactively seek out areas
for improvement in the client application portfolio. One way to determine
whether there is scope of improvement in an application in the client IT
landscape is by benchmarking its performance against other similar applications
in other client landscapes. To illustrate, suppose that a service company
maintains the payroll applications for 10~clients. For nine of the clients, the
monthly ticket volumes range between 5 and 10 tickets, whereas, for the
remaining client (say client~A), the ticket volume ranges from 20 to 50. This
indicates that, for client~A, investigations could be conducted to determine the
root causes for high ticket volumes and appropriate preventive actions
taken. Moreover, if a similar high-ticket-volume problem was seen in the past in
another client's payroll system, information about the actions taken for that
client could help the team resolve the problem for client~A.

There are a number of organizations that gather and report quantitative
benchmark information (qualitiative as well as productivity) for software
projects and business applications depending on language used to code, number of
users etc. CAST, Software Engineering Institute are some examples of such
organizations. However, considering services organizations are doing multiple
software implementations for multiple clients, an organization-wide knowledge
base that (1) captures key operational metrics per application (or at a more
granular level, such as by problem area) per client and the past improvement
actions taken, and (2) allows comparisons between clients with similar
applications and problems encountered, would help augment what is publicly
available and be more useful.

\begin{figure*}
	\center
	\includegraphics[scale=0.4]{figs/km.png}
        \vspace*{-10pt}
	\caption{System architecture for an OKMS system for a service company.}
        \vspace*{-10pt}
	\label{fig-km}
\end{figure*}

\subsection{Research Topics}

Based on such OKMS needs in IBM's service-delivery organization, there have been
efforts toward implementing systems to address the needs. These include a system
for promoting solution reuse in design-build engagements centered around
business-process transformation~\cite{Goodwin:2012b}, and a system for sharing
information about problem tickets across client
engagements~\cite{Majumdar:2011}.

We next discuss some core problems that need to be addressed in creating OKMS in
service organizations. At the simplest level, an OKMS is a database where
content can be stored and retrieved from. However, what content should be
stored, how easily can the content be stored, and the ease with which relevant
content can be retrieved determines the usefulness of an OKMS.
Figure~\ref{fig-km} presents several activities (shown as block arrows) that
need to be performed in creating and using an OKMS; for each activity, the
figure displays relevant topics (shown in the boxes) that would benefit from
further research. The activities are broadly classified into two categories:
knowledge creation and knowledge retrieval.


\subsubsection{Knowledge Creation}

What constitutes useful knowledge in a service company? How can such knowledge
be collected and organized? These are some of the questions that need to be
addressed in knowledge creation. Specifically, we discuss three aspects of
knowledge creation: \textit{crawling}, to collect content from diverse
sources; \textit{parsing}, to translate the content in various formats into a
standard format; and \textit{annotating}, to extract metadata from the content
that can help organize it.

\vskip -5pt
\paragraph*{Crawling} The first activity in building an OKMS is
identifying the data that should be stored in the repository and how to obtain
the data. In general, this activity includes a combination of manually provided
data and automatically crawled data, each of which can have its own peculiar
challenges.

For manual data collection, an organization can require its employees to
contribute learnings and software artifacts to the OKMS. For instance, knowledge
creation can be done via frequently asked questions, where employees outline the
solutions to some common problem tickets they have resolved; alternatively, it
could be done via postmortem reports, usage stories, and experience
reports~\cite{desouza:2005} created by project members after the resolution of a
key project issue. If a client solution involves software development, employees
can be asked to identify reusable components in the software and contribute
them, possibly after generalization, to the OKMS. Manual content creation puts
extra burden on the employees, beyond their normal delivery-related
responsibilities. Thus, an interesting research problem is how to motivate
employees to contribute high-quality content to the
OKMS~\cite{hendriks1999share}. Incentive mechanisms could include ``badges'' as
is done in open-source forums such as stack exchange to encourage question and
answer contributions. In a service company, would reputation-building incentives
be sufficient or would monetary or career-growth incentives be
necessary~\cite{bartol2002encouraging}?

%% An organization has the option of mandating that each of it's employees
%% contributes their learnings to the OKMS system. Some examples of manual
%% knowledge creations are: (1) frequently asked question, where employee outlines
%% the typical solutions to some common problem tickets they have resolved (2)
%% postmortem reports, usage stories, experience reports\cite{desouza:2005}, that
%% project members can create after they solve a key issue in the project. To
%% enable solution reuse, services organizations often invest in creating software
%% product families
%% \cite{clements2002software}. However, here it's pre-anticipated what could be
%% some solutions that could be of interest to multiple clients in a particular
%% domain such as healthcare and then those solutions are created with appropriate
%% points for variability built in, so solution can be customized for any client
%% intending to use it.  Once a project is completed, employees are also encouraged
%% to identify reusable components in software they created and share them in the
%% knowledge base. This approach of manual content creation adds extra burden on
%% the employees. An open research challenge is to motivate employees to contribute
%% high quality content in the knowledge base \cite{hendriks1999share}. Open source
%% forums such as stack exchange have experimented with elaborate incentive
%% mechanisms in forms of badges to ensure questions and answer contributions on
%% their forums \cite{vasilescu2014social}. In services organizations are
%% reputation building incentives enough or incentives should translate to monetory
%% benefits and/or career progression /cite{bartol2002encouraging}?

In addition, or as an alternative, to manual content creation, content can be
collected through automated crawling of the artifacts produced in a project. For
example, complete traceability from requirements through code to test cases,
along with relevant content, can be extracted from Application Lifecycle
Management tools (\eg Rational Team Concert). Similarly, information about
problem tickets and their resolutions can be extracted from ticket-management
systems.  Account teams periodically report important metrics, such as ticket
volumes, code changes, and service-improvement actions taken, on the client
applications being maintained; such reports can be automatically pushed into the
OKMS. The main challenge here is handling the diversity of tools and
technologies used to create software artifacts and related information across
projects.

%% Another approach for content creation is to auto-harvest artifacts produced in
%% the SDLC lifecycle and put these in the knowledge management system. Complete
%% traceability from requirement through code to test cases, along with their
%% content is extractable from Application Lifecycle Management tools and this can
%% act as solution packs to put in the repository. Similarly for ticket resolution,
%% information about problem ticket and it's resolution is extractable from ticket
%% management systems. Account teams periodically report on various important
%% metrics on the applications being maintained in the client landscape such as
%% ticket volumes, code changes, service improvement actions taken. These reports
%% can be auto pushed into the knowledge repository. The main challenge here is
%% handling the diversity of tools and technologies used to create SDLC data across
%% projects.

\paragraph*{Parsing} Once all the requisite data has been pulled from various sources, the main challenge is parsing of the data from various document formats and templates into a standardized format that can be pushed into the knowledge repository. Many of the crawled data is in form of documents in proprietery formats such as Microsoft word, power point slides, visio diagrams, excel sheets. How to automate content extraction from client specific work products into the standard format used by knowledge repository is again a direction for research. One approach is to write model to model transforms where a project admin can specify the mapping \cite{debdoot:2010:scc}. Another challenge is that requisite traceability information that is required to understand complete context of how and why an artifact was produced, might be missing. This is because tools being used to create different SDLC artifacts have not been integrated. There is need to auto infer traceability between artifacts
For example, if making a code commit, the developer puts in a comment like "Fixed Bug \#145", then with high confidence the change is to fix "Bug \#145" and traceability edge between code file and bug report should be auto-created. 
. Tracability inference is an active area of research \cite{spanoudakis2005software}. Some proposed approaches use information retrieval (IR) techniques, others use traceability rules, special integrators, and inference axioms.

\paragraph*{Annotation}

The content put in the OKMS should be organized in such a way that retrieval becomes easier when someone needs it. The general practice is to classify OKMS content against pre-defined taxonomies. One taxonomy is obviously the object type i.e. requirement, code, problem ticket (further segmented into problem description, resolution). Another taxonomy captures the domain the artifact was produced in i.e. industry type and process area e.g. \cite{apqc,bph}. Another taxonomy is the technology used. Organizations spend effort building and maintaining these taxonomies. For every data that is put into the knowledge base, the content needs to be manually categorized against these pre-defined taxonomies. Pattern matching and machine learning based classification approaches /cite{bishop2006pattern} can be used to auto-categorize content to these pre-defined taxonomies. These techniques rely on the availability of equal proportion of positive and negative samples to train a learning model. However, due to unavailability of training data and/or lack of differentiating features, usual learning techniques such as naive bayes, support vector machines, decision trees, might end up not giving desired efficacy (measured as precision and recall). There is need to customize more advanced learning approaches such as adaptive learning, ensemble techniques or develop new techniques that work well with SDLC data. Another interesting area for research is to explore how to help grow the taxonomy over time based on content coming in the knowledge base. Techniques such as clustering \cite{Berkhin06}, topic modeling \cite{Blei:2012} help group together similar looking content and infer topics out of them. 

The content in a OKMS database comes from various clients:. There are strict privacy constraints around what data is client confidential and hence cannot be shared. Within a single artifact itself, there might be small portions of content that are client confidential. E.g. a problem ticket might content the contact information of the user who encountered the information. How to remove client confidential data from artifacts put in the repository, how to anonimize the content so as to not disclose client identity and how to ensure that only authorized users and roles have access.

Once an OKMS system is implemented in an organization, irrespective of the approach to collect content i.e. manual, automated harvesting or hybrid, the repository starts filling up fast. Over a period of one year, the problem ticket repository we setup within IBM collected 750K tickets. Similarly, the business process solution repository has 16000 solutions. However, not all content being put in the repository is high quality and reusable. Hence, it is becomes imperative to be able to filter out useless content. One way to achieve this is by making a human vet every content being pushed in the repository and only content that is deemed high quality is published. Another approach is to ask people who are retrieving and potentially using the content, give feedback on whether they found content useful or not. The third approach that makes for an interesting research direction is to explore how a an automatic quality score can be assiged to each artifact based on content in the artifact, prior reputation of people who authored the content, whether the project was a success or not and so on. In our problem ticket repository, we experimented with calculating a quality score per ticket based on technical versus non-technical content present in the ticket \cite{Majumdar:2011}.


\paragraph*{Summary} To summarize, content creation in OKMS provides multiple oppurtunities where research can contribute. These include: what and how to extract content from SDLC repositories and proprietry document formats, how to infer traceability between different artifacts, how to classify and categorize content, how to maintain privacy, estimate quality and motivate employees to contribute high quality content. 



\subsubsection{Knowledge Retrieval}

Challenges in knowledge retrieval are related to the search capabilities offered
by the OKMS and the effort required from users in determining the usefulness of
the recommended data. We discuss the following aspects of knowledge retrieval:
indexing and searching, search result organization, and advanced analytics.

\paragraph*{Indexing and Searching} The simplest approach for searching
is \textit{keyword-based search}, in which users specify the words they are
looking for and the system returns all artifacts that contain the specified
words. More sophisticated systems use \textit{faceted search}, where the user
can navigate a hierarchical structure (\eg a taxonomy) and select values from
predefined categories.  Despite the availability of many search technologies,
studies (\eg \cite{idc,idc2}) have shown that retrieval of relevant information
from organizational repositories remains challenging, with users being
successful in less than 50\% of their attempts in searching for information.

Language-based information-retrieval techniques~\cite{manning2008introduction},
on which many knowledge systems are based, may be inadequate in dealing with the
types of repositories we are talking about---that contain not just a collection
of artifacts, but a network of linked artifacts.  Graph databases for storage
and semantic search techniques~\cite{Guha:2003} or extending keyword search for
graphs~\cite{kacholia2005bidirectional} may be more promising in this scenario,
and are worthy of research investigation.

%% \paragraph*{Index and Search} Indexing is how the knowledge repository stores
%% data internally. Search features then work on this index.  Typical ways to
%% retrieve content from a knowledge base are: (1) keyword based search where user
%% specifies a couple of words (s)he is looking for and all artifacts in the
%% repository that contain the words from user query are returned, (2) faceted (or
%% navigational) search where user is shown a hierarchy structure (taxonomy) and
%% can browse information by choosing one or more values from each of the
%% pre-defined categories. Various language based information retrieval
%% models \cite{manning2008introduction} such as vector space models, probablistic
%% models, latent semantic index exist that can be used here. But inspite of easy
%% to use search technologies being available, prior studies report that knowledge
%% retrieval from organization wide repositories remains a challenge. As
%% per \cite{idc,idc2}, while employees spend 15\% to 35\% of their time searching
%% for information in an enterprise, they are successful less than 50\% of the time
%% in finding what they are looking for. Most existing OKMS systems use language
%% based IR models to store and retrieve knowledge. However, as we saw in content
%% creation section, the repository is not just a collection of artifacts but a
%% network of linked artifacts. Use of graph databases for data storage and
%% retrieval techniques such as semantic search techniques \cite{Guha:2003} or
%% optimizing keyword search for graphs \cite{kacholia2005bidirectional}, are worth
%% exploring.

An important factor in performing accurate search is the expressiveness of query
construction. Simple keyword queries, consisting of a set of words, may not be
discriminative enough to return accurate results. A few approaches have been
developed to address this problem, for example, by giving more weights to words
that appear in the title of a problem ticket than words that appear in the
ticket description~\cite{Sinha:2012}, and using separate queries for different
types of information, such as description, application information, and stack
trace, in a problem ticket~\cite{Ashok:2009}.

Another question pertains to assigning weights to clauses in the query while
ranking the search results~\cite{Debdoot:2011:bpm}. Moreover, the idea
of \textit{contextual search}~\cite{wen2004probabilistic,kraft2005q}, which
attempts to capture the user's information needs better by augmenting the query
with contextual information extracted from search context, has shown promising
results. Further research along these directions---in the context of the
knowledge needs in service delivery---investigating different ways of creating
contexts, using contexts in constructing queries, and assigning weights to query
clauses would be interesting.

%% Non-availability of content or poorly organized content can be one reason for
%% this. Another reason could be the inadequacy of the query itself that are used
%% to retrive the content. Suppose a user is trying to find problem tickets that
%% resolved similar issues to what (s)he has been assigned. (S)he would pick up a
%% couple of words from the ticket that describe the problem and use it to query
%% the knowledge base. However, these words might not be discriminative enough and
%% user might end up getting too many or too little search hits. Another approach
%% could be to use complete content in the ticket and use it as query. However in
%% this case, the search engine might end up returning irrelevant results as equal
%% weightage was given to all words in the problem ticket. \cite{Sinha:2012} tries
%% to address this issue by giving more weightage to those words in the query that
%% belong to ticket title. \cite{Ashok:2009} parses out different datatypes from a
%% problem ticket such as description, application information, stack trace and
%% using a different query/search mechanism for each. E.g. instead of just
%% specifying "null pointer exception", the query generated from a problem ticket
%% could be---description: contains following bag of words \{null, pointer,
%% exception\}, process: is equal to "order to cash", stack trace: contains
%% foo.*bar. The results obtained from such a query are likely to be more precise
%% than what a keyword search would yield. A challenge with this approach is how to
%% compose the various clauses in the query---are they "anded" or "ored" or a
%% combination. Another challenge is how much weightage to give to each clause when
%% ranking the search results \cite{Debdoot:2011:bpm}. Moreover, recently
%% contextual search \cite{wen2004probabilistic,kraft2005q} has been gaining
%% momentum. Contextual search tries to better capture a user's information need by
%% augmenting the query with contextual information extracted from search
%% context. It would be an interesting direction for future research to see for
%% each of the knowledge needs in services organizations, what can make up the
%% context, how to use this context to create the query and how to weigh different
%% clauses in the query.

\paragraph*{Search Result Organization}
Although more powerful querying capabilities can help improve the accuracy of
search results, a search-based system would in general return multiple
results. The next question then is: how much effort does it take for the user to
sift through the results to find relevant information? Simple approaches such as
highlighting matching keywords will not work for complex queries; more
sophisticated techniques that help the user easily comprehend the relevance of
the results are needed. One such approach is \textit{summarization}, which has
been used in text-processing domains to end users get a quick overview of
information; summarization has also been applied to bug
reports~\cite{Mani:2012,Rastkar:2010}. Extending this notion to other types of
artifacts, such as code, requirements, and solution designs, is an open topic
for research.

Another types of analysis that can improve the search results is detection and
grouping of duplicates or near-duplicates. By grouping together such artifacts
and highlighting the variances in seemingly similar artifacts, the system can
reduce information overload on the user.  Detection of duplicate bug reports has
been widely researched
(\eg \cite{wang2008approach,sun2010discriminative}). Developing such analyses
for other types of artifacts, from a services OKMS perspective, would be useful.

%% \paragraph*{Search Result Organization}
%% Any search based system would return multiple results. For the user it becomes a
%% chore in itself to go over each of the search results and identify if it is
%% relevant or not. Techniques that can help the user easily comprehend the
%% relevance of the returned results are much needed. Text based search engines
%% highlight the matching words between query and content of the artifact
%% returned. However, when the query becomes complex (as above), keyword
%% highlighting is not of much help. In text processing domain, summarization is
%% one approach that is used to help end users get a quick idea of what the content
%% is about. In \cite{Mani:2012,Rastkar:2010} authors present different techniques
%% to summarize bug reports. Research can help in developing summarization
%% techniques that work for other SDLC artifact such as code, requirements,
%% solution design documents and so on. Further, out of the search results returned
%% many artifacts could be duplicates or near duplicates of each other. In order to
%% reduce information overload for end user, the OKMS system should be able to
%% group together duplicates and near duplicates and be able to highlight the
%% variances in seemingly similar artifacts returned in the search
%% results. Duplicate bug report
%% detection \cite{wang2008approach,sun2010discriminative} has been a widely
%% researched topic. Similarly for each of the artifact of interest from a services
%% OKMS perspective, a customized duplicate detection approach might be needed.

\paragraph*{Advanced Analytics}
Most knowledge-management systems today stop at providing search
capabilities. Given the magnitude of data that can be collected in OKMS, there
is the opportunity of implementing advanced analytics that go beyond by
supporting decision making. Consider the scenario of troubleshooting where, to
resolve a ticket, the user searches through past resolved tickets to find
similar tickets.  Based on the past tickets and the current context, the user
has to formulate potential hypotheses about the root cause, decide which
hypotheses are applicable, and then investigate the solutions. The OKMS would be
much more useful if it could automatically generate potential hypotheses about
root causes and solutions for the user to investigate. Such analytics could also
be developed for other scenarios such as service improvement.

%% \paragraph*{Advanced Analytics}
%% OKMS today stops at providing search capabilities. Given a user query, the
%% system returns back matching artifacts but makes no attempt at making any
%% deductions. Consider the scenario of troubleshooting. A user is searching
%% through the past resolved ticket repository to see if similar issues were
%% resolved. There are multiple reasons why the issues could have arisen. Based on
%% past tickets, the user needs to come up with potential hypothesis of why the
%% issue could be arising and based on conditions (s)he is seeing in the current
%% landscape decide what hypothesis is applicable and then pick the appropriate
%% solution. From such a user's perspective the OKMS system would be more effective
%% if it were able to auto-generate these potential root causes and solutions
%% hypothesis. Consider the service improvement use case, the accout team needs
%% support from OKMS to identify similar projects, then compare the problems seen
%% in the current project with different kind of issues arising in similar
%% projects, then judge whether current team is doing better or worse and then
%% decide to take some action. What kind of capabilities are needed in OKMS system
%% to be useful in above scenarios is another area for research.

In general, there are two aspects of organization knowledge management:
codification and personalization~\cite{hansen2000s}. Our discussion of OKMS has
so far focused on the codification aspect, where knowledge is carefully captured
and stored in a knowledge base for use by anyone in the company. The
personalization aspect ties knowledge to specific people; the main purpose of
knowledge management here is to help the user find and connect with people who
have the requisite knowledge, not store the knowledge. This strategy is
especially appropriate in cases where knowledge cannot be easily codified, such
as the knowledge for handling situations that require complex decision
making. Expertise browsing, recommendation, and mining have been explored in
various contexts (\eg \cite{Balog:2006, McDonald:2000, Mockus:2002}). Extending
these ideas to the context of service delivery, where the OKMS contains data
from multiple projects in different contexts, is an area where further research
can help.

%% There are two techniques for organization knowledge management, codification or
%% personalization \cite{hansen2000s}.  Till now what we have discussed is what is
%% called codification. Here knowledge is carefully captured and stored in the
%% database, where it can be accessed and used by anyone in the company. This
%% strategy allows many people to search for and retrieve knowledge without having
%% to contact the person who originally developed it. This opens up the possibility
%% of achieving required scale in knowledge reuse in services
%% organizations. Another strategy for knowledge management is
%% personalization. Here knowledge is tied to person who developed is and is shared
%% mainly through direct person-to-person contacts. The chief purpose of knowledge
%% management system here is to help people find and connect to other people who
%% have requisite knowledge, not to store it. This strategy works well when
%% knowledge cannot be codified especially knowledge required to handle situations
%% that require complex decision making. Expertise browser \cite{Mockus:2002},
%% expertise recommender \cite{McDonald:2000} have attempted to identify experts on
%% various topics within a project. \cite{Balog:2006} has explored expertise mining
%% and recommendation in predominantly document based OKMS systems. What
%% information to use to mine expertise when repository contains SDLC data from
%% multiple projects, how to match your current context and job profile to suggest
%% people you should have in your network is another scenario where research can
%% help.

%% \paragraph*{Summary} To summarize, content retrieval in OKMS system provides
%% open research problems in query generation and use of context to augment
%% queries, search result summarization, duplicate detection, use of semantic
%% search to improve search relevance. Further the retrieval capabilities need to
%% move beyond just search to provide capabilities such as question-answer
%% generation, cross-account comparison/benchmarking and expertise recommendation.



%\section{Troubleshooting}

One of the common forms of service engagements is application maintenance, where the expectation from the service provider is to take over a client's custom application and handle service requests for it.  Service requests come in the form of trouble ``tickets'': users of the applications can raise a ``ticket'', logging a problem they have been experiencing. This is similar to bug repositories such as Bugzilla, in which users of an open-source software can enter the details of a problem that they experience.  The main difference is that in typical software development in open-source communities, there is no obligation on the part of software maintainers to address the defects in a timely fashion. Likewise, even in a product setting, the development team can prioritize which defects they are going to address first.  In service context, the service provider is supposed to resolve the ticket in a timely manner, often under a service-level agreement. For example, a critical bug must be resolved within 6 hours, at risk of financial consequences for the provider. 

Clearly, the efficiency which the service provider can resolve these tickets is a differentiating factor. For this reason, troubleshooting is a important research area for software services.

Ticket resolution takes place in stages. At the front, customer facing level is what is called L1 support, which simple acknowledges the issue to the customer and assigns it to an internal queue. In simple cases, such as request for information, L1 support can provide ``help desk'' features. L2 support comes in when the ticket requires technical expertise to answer, but usually resolving the ticket entails no more than advising the customer of the right configuration, or a workaround, etc. Finally L3 support handles real defect, and resolving them typically involves fixing the code.

(We can use a picture here.)

L1 and L2 support can benefit from the techniques that we reviewed in the section on knowledge management.

Here we describe some of the challenges in L3 ticket resolution.

In the purest form, the troubleshooting problem (a.k.a. debugging) is this: 
\section{Risk identification and management}


Software risk management is a well established discipline~\cite{risk1,risk2} that has generated continued academic interest as the complexity and nature of the software projects have evolved over time. Traditional risk management techniques have been focused on identifying and codifying best practices that prevent or reduce the failure rate~\cite{risk3,risk4,risk5,risk6,risk7}. Large IT organizations have assimilated many of these findings in their risk management practice.  However, adopting these best practices does not guarantee that risk is eliminated or even reduced to an acceptable level -- new software development models driven by globalization, competition and an ever changing software landscape throw up new patterns of trouble.

In recent literature, the trend towards a systems approach to risk management, where in statistical techniques are used to classify and mine the project metrics data, with a view to pro-actively learn the new trends is gaining acceptance~\cite{risk8,risk9,risk10}.  These methods actively use the metrics collected during traditional risk management reviews and then employ techniques borrowed from statistical learning theory to derive models that describe the relationship between the collected metrics and eventual project outcomes. Thus, certain patterns of values in project metrics can act as alerts, which can be then used to initiate further reviews to see whether the alert was justified. It is important to note that these techniques in most cases cannot identify why a project may be troubled i.e., they can only serve as a starting point for further investigation.  Another significant short coming of these models is their inability to implicitly account for outliers and missing data -- for example, a project in good health will not typically have as thorough a review as a project in trouble9. Unless the data that is used to train the model is carefully validated to account for these characteristic patterns, there is a good chance that the model will not be very accurate. This problem is particularly prevalent in large IT delivery organizations which typically exhibit a large variance in both the quality of the metrics collected and reporting patterns.

Functionality risk is defined as the risk that the completed system will not meet its user's needs~\cite{risk11}.  It is the risk that churn in business requirements and scope combined with ineffective communication mechanisms leads to the development of working software that is essentially useless to the end user~\cite{risk12}. In a recent article~\cite{risk13}, it has been argued that functionality risk has become an increasingly important factor due to the challenges  that globalization has created. Through empirical analysis, they identify that development methodology fit, customer involvement, and use of formal project management practices as the top three functionality risk factors. In complex IT engagements we find that reducing or eliminating functionality risk, which in turn leads to high customer satisfaction, may result in an unacceptable increase in other risk factors such as financial risk for the provider. Thus, in our approach we opt to look for trouble patterns across a broad spectrum of risk indicators and surface problems which then allow risk managers to make informed decisions.

Quantitative analysis of IT investment decisions using options analysis, to mitigate financial risk, is an active area of research~\cite{14,15,16}. Options analysis can be used to assess the value of prototyping work and early adoption initiatives related to new IT platforms – options provide a way to evaluate the value of IT projects which give the right to adopt the resulting technology without having the obligation to do so. Fichman~\cite{risk14} shows how options analysis can be used to predict IT platform initiation and adoption, value IT platform options and manage IT platform implementation. Chen and Sheng~\cite{risk15} discuss how to do real options analysis while accounting for estimation errors. However, the ability to use the results of the analysis in a dispassionate manner to start and cull IT projects is still a contentious topic~\cite{risk17}. As Keil et al.~\cite{risk18} observe, there is a strong bias in many organizations to continue with IT projects even though financial analysis indicates otherwise. In our view, the lack of widespread adoption for options analysis can be traced back to a fundamental issue that effects all methods for analyzing financial risk – the inability to accurately estimate with any degree of certainty the net present value (NPV) of a complex in-flight IT project. In this paper,  we evaluate financial risk by analyzing how projects that exhibited similar trends in project metrics performed in the past – thus, the analysis has no dependence on a particular methodology for evaluating NPV.

Evaluating IT project portfolios with a view to making strategic decisions that help the portfolio grow in accordance with business needs helps target limited budgets on relevant projects. Armour~\cite{risk18} argues that a company’s IT project portfolio should be treated like any other investment portfolio to understand whether the reward justifies the estimated risk. Given the uncertainty surrounding the estimation of risk, Lin~\cite{risk20} suggests that fuzzy logic may be used to make IT portfolio decisions. Holland and Fathi~\cite{risk25} suggest that uncertainty can be reduced through portfolio diversification and spreading the risk of incorrect risk management decisions across a wide pool of projects. Our approach to surfacing trouble indicators provides tools for a company to make strategic decisions about its IT project portfolio.

Erickson and Evaristo~\cite{risk22} provide an account of how risk factors associated with IT projects are magnified or multiplied when dealing with distributed project teams.  They provide a conceptual list of a variety of factors ranging from culture to distance and discuss how an increase in distributedness along these dimensions affects project risk.  Ramasubbu and Balan~\cite{risk23} present an empirical study which quantifies the loss in quality and increase in schedule risk that one would expect in global software projects. They suggest that good software process controls may mitigate these losses to some degree. Beise~\cite{risk24} suggests that good project management practices when properly applied may help to mitigate some of the problems caused by virtual teams. Our analysis suggests that even under the best possible control structure the losses due to distributed development are unavoidable -- the ability to recognize the symptoms and act quickly will decide which  companies will succeed in leveraging the promise of globalization.

It is well known that decision-makers, including those in the risk management community, routinely use flawed heuristics in decision-making, which are subject to systematic biases~\cite{risk27}. Heemstra~\cite{risk24} analyzes how filters and biases of personnel involved in judging a project may prevent an accurate risk assessment. He suggests a team based decision making approach to avoid being heavily influenced by particular individuals. Host and Lindholm~\cite{risk25}  do an empirical analysis of how a set of identified project risks are weighted by different individuals. They find a wide variance in the perceived importance attached to individual risk factors -- however, they could not explain this variance on the basis of the particular role played by the person in the project. Maytorena~\cite{risk26} provides an interesting study on the effect of experience on the ability of project managers to recognize certain risk factors. It shows that experience does not have a significant impact on the ability to detect risk as do other factors like risk management training, ability to quickly grasp information and the level of education. In practice, we find that packaging the statistical information and trouble patterns in an easily understandable and actionable fashion to the risk managers may be as important as the information itself.

\subsection{Risk prediction}
 In any complex software development project, a number of risk assessment related activities are conducted prior to project inception with an objective of determining the risk entailed in delivery. The number and depth of these reviews is determined by a variety of factors, such as: the size of the project, the novelty of the proposed solution, and the industry sector to which the client belongs. These reviews may cross multiple delivery organizations within and beyond the purview of the primary service provider who is responsible for the delivery of the integrated software solution. Even for a large services provider that performs thousands of software projects with a variety of clients, it is not clear that there is infact a common set of criteria to judge the risk involved in these projects due to the great diversity in the projects and the client context in which they are executed. In this section, we will describe an approach to this problem of risk identification which has proven to be highly reliable and has been thoroughly tested in the field over the last five years.

 By reviewing the risk management reviews of several recent IT delivery projects, we identified a set of sixteen questions that were considered to be predictive of the project outcome in the initial phases of the project life cycle. These were then whetted by experienced risk managers and carefully screened to ensure that they can be used to obtain clearly defined answers within well-defined ranges. The intent was to remove subjectivity in the answers by basing them on data that is readily accessible to the risk manager.  Questions ranged from a client's past experience with the delivery organization, match of the delivery team’s technical skills to the project objectives and the type of contract (fixed price vs. time and materials).  A key point to be noted is that these questions do not delve deeply into the technical details of the project itself, as they need to cover a variety of IT projects ranging from implementing a custom application to customizing a packaged application. However, all the questions have proven to be highly correlated with the eventual outcome of the project.

 The traditional risk management approach was to scale the answers for these set of questions for a given project, sum the scaled answers and apply threshold(s)/range(s) on the result which can be then associated with a given outcome of a projected future project management review based on correlation with historical data. However, the pitfalls in such an approach are clear – dependencies/interactions between sets of questions may lead to inconsistent results over a broad set of projects. For example, a “fixed price” deal combined with a novel (first of a kind) technical solution may be more risky to undertake than a “time and materials” deal in the same context. Figure~\ref{riskfig1} shows the relationship between a simple scaled average and the project management review letter grade of over 130 real software projects (in this example, project risk significantly increases as the letter grade increases from A to D). It is clear that the choice of any given set of thresholds would lead to a misclassification of a number of projects.

 <<Risk Figure 1>>

 We took an alternative approach where we used statistical classification algorithms to match the risk management questions with the eventual project outcome~\cite{risk28}. Prior to classification, the answers to the risk management review questions were scaled and binned appropriately to reduce the effects of variance due to human error and differences in procedures followed in different countries. For example, a question about the number of customer contacts prior to the signing of the deal was reduced to three classes – high, medium and low rather than asking for an exact number (since the definition of a customer contact can be interpreted and even tallied differently by different risk managers). We chose to apply a decision tree classifier to the data set, as this approach not only produced acceptable results, but the resultant rules make the prediction process transparent to the end users. We split the available data set into a training and test data set through random sampling. Figure~\ref{riskfig2} shows the cost and misclassification matrices for this approach. The cost matrix allows the risk managers to indicate the relative weight to be associated with a particular type of misclassification. For example, the cost matrix shows that misclassifying a project that would have been a C as a D project carries a weight of 1, where as a more serious misclassification of C as an A carries a weight of 9. As can be seen, false positives (good projects being classified as troubled) carry less weight than false negatives (troubled projects being classified as good). The classification matrix shows how each category of projects fared in terms of prediction. As expected, the number of false negatives (3\%) is less than the number of false positives (7\%) and more than 94\% of the troubled projects (C/D) are captured by the classification scheme.
 
<<Risk Figure 2>>


\label{sec:risk}


\section{Research Direction: Testing and Troubleshooting}
\label{sec:testing-debugging}

\subsection{Software Testing as a Service}

Within the increasing demand for software development as a service,
\textit{software testing as a service} has seen significant growth and adoption
in its own right, often as a separate service that is independent of development
activities. In fact, according to a 2006 survey,\footnote{\scriptsize
  \url{http://www.drdobbs.com/architecture-and-design/cheapers-not-always-better/184415486?requestid=247829}}
software testing was the second largest outsourced software-engineering activity
after coding: 81\% of the 200 industrial practitioners, who participated in the
survey, stated that they outsource software testing. Given this trend, services
companies now routinely offer services that focus exclusively on testing
activities.

%% The engagement modes can vary: staff augmentation, core/flex, managed service
%% (fixed capacity, outcome based). But perhaps this needs to be mentioned earlier
%% as these modes are not particular to testing services.

The nature of the activities performed in a testing-services engagement can vary
from client to client; but, typically, the scope of work includes activities,
such as test design, test-data creation, test automation, test execution, and
test maintenance. These activities, of course, pertain to testing in general
(whether performed in-house or in an outsourced manner), but there are factors
that can add unique challenges in the setting of testing services. For instance,
many of these activities can require specific skills in testing techniques/tools
or coding expertise, which the average practitioner involved in service delivery
may not possess---at least not in the myriad of technologies 

%% Limited control over test execution environment

Research Problem: Efficient and change-resilient test automation without good
programming skills or deep tool knowledge (e.g., DOM representations in
tools). Maintenance of test scripts by non-developers.

No one technique of UI element identification is likely to be the best in all
circumstances; therefore, the choice of the technique needs to be based on the
requirements for testing (e.g., execution of scripts across browsers, across
internationalized variants of application, etc.)

Cross-browser testing; keeping up with browser upgrades, reduce reliance on
DOM structure and attributes.

Research direction

Emphasize links to program synthesis: (1) basic generation of change-resilient
method of identifying a UI element; (2) synthesis of custom functions based on
some specification of user intent.

Problem of test automation in the mobile app space. More diversity and
fragmentation. Bigger problem: UI differences across app variants.

Research Problem: Test planning and test design

Research Problem: Test data provisioning. Privacy issues, dependence on client.

Research Problem: Test management and optimization


Work done in IBM, work done by others (in industry and academia), where do we
see this topic going

\section{Experiences}

Over the past few years, we have worked extensively on taking research innovations from the lab to real-world deployment in IBM Global Business Services. In this section, we highlight some of our experiences.

First of all, there is a fundamental tension between client expectation of the service provider being bold and innovative, and at the same time provide predictable, consistent delivery.  When vendors bid for a contract, the client does want the vendor to bring innovation to impact both cost and quality. Once the project gets going, however, the ground level realities of consistent delivery must be met, and therefore project managers in charge of delivery need to be conservative in their approach.

Any new process or tools causes at least a minor disruption in normal flow of the project, and this makes pushing either a new process or a new tool into project delivery significantly harder.  

To compound matters, a project may move from one vendor to another vendor for various reasons, and for this reason, the client may not want to deviate from ``industry standard'' processes and tools, because it would be difficult to back out of any thing experimental. Clients often prohibit vendors from installing additional software in their IT environment.

The second challenge is that the skill levels of the practitioners vary widely, and moreover, business pressures often requirement to move from project to project. In such a scenario, getting practitioners to enthusiastically learn new tools is difficult.  Moreover, tools in which there is no immediate reward to them --- e.g. entering information in a knowledge sharing system --- are even harder to get adopted.

The third challenge is that services companies are not equipped to curate software tools in house. Thus, a tool developed in the labs, even when it proves its value, sometimes meet with difficulties because there is no organization to ``own'' and support the tool for long periods of time.

The research challenge is to design tools in a way that are (a) minimally disruptive and (b) easy to learn. Incentivization schemes, also known as gameification, are an exciting area of research with tremendous implications on service delivery.  Open sourcing may be a way to get tools to be supported in the community for a long time, though this obviously is at odds with vendors desire to have exclusive, differentiating technology.


\section{Conclusion}
\label{sec:conclusion}

Software services companies rely crucially innovations in individual and
organizations productivity for their competitiveness. In this paper, we examined
some of the topics that require attention from the software engineering research
community. We examined challenges in how modern services companies organize
their work and workforce; challenges in knowledge management at an organization
level; issues in risk assessment; and finally challenges in testing services. We
also discusses challenges in deployment of research innovations in real-world
services context. We hope that the research community takes a careful look at
the software services industry for many opportunities for impact.

\section*{Acknowledgements}
The research agenda presented in this paper is a collation of knowledge the
authors and various other IBM Researchers accumulated in course of working with
IBM's service delivery practitioners. From services delivery, we would like to
thank Anup K Ghosh, Jack Bisceglia, and Achin K Das for sharing the IT delivery
challenges and championing the use of research tools in their teams. At IBM
Research, we would like to thank Sugata Ghosal, Manish Gupta, and Rakesh Mohan
for driving the services delivery research agenda. We would also like to
acknowledge our colleagues Min Chee, Pankaj Dhoolia, Richard Goodwin, Juhnyoung
Lee, Debapriyo Majumdar, Senthil Mani, Pietro Mazzoleni, Debdoot Muk\-herjee,
Diptikalyan Saha, Karthik Sankaranarayanan, Bikram Sengupta, Renuka Sindhgatta,
Suresh Thummalapenta, and Joe Zhou, who have been involved in various research
initiatives during course of which the challenges listed in this paper were
identified.

\bibliographystyle{abbrvnat}
{\small
%\balance
\bibliography{fose,risk-global,km}
}

\end{document}
